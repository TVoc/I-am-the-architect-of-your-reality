\documentclass[a4paper,10pt]{article}

\usepackage[english]{babel}
\usepackage{graphicx}
\usepackage[colorlinks, linkcolor=black, citecolor=black, urlcolor=black]{hyperref}
\usepackage{geometry}
\geometry{tmargin=3cm, bmargin=2.2cm, lmargin=2.2cm, rmargin=2cm}
\usepackage{todonotes} %Used for the figure placeholders

% Your name and student number must be filled in on the title page found in
% titlepage.tex.

\begin{document}
\input{titlepage}

\tableofcontents
\newpage

\section{Domain analysis}\label{sec:domain}
\subsection{Domain models}
This section shows the domain model(s).

\begin{figure}[!htp]
    \centering
    \includegraphics[width=0.8\textwidth]{domain_model.png}
    %\missingfigure[figwidth=0.8\textwidth]{Domain model}
    \caption{The domain model for the system.}\label{fig:domain_model}
\end{figure}

\subsection{Domain constraints}
In this section we provide additional domain constraints.

\begin{itemize}
    \item This is a first constraint.
    \item This is a second constraint.
\end{itemize}

\subsection{Glossary}
In this section, we provide a glossary of the most important terminology used
in this analysis.

\begin{itemize}
    \item \textbf{Term1}: definition
    \item \textbf{Term2}: definition
\end{itemize}

\section{Functional requirements}\label{sec:functional}
\subsection*{Use case model}

\begin{figure}[!htp]
    \centering
    %\includegraphics[width=0.8\textwidth]{}
    \missingfigure[figwidth=0.8\textwidth]{Use case model}
    \caption{Use case diagram for the system.}\label{fig:use_case_model}
\end{figure}

\subsection{\emph{UC7}: Filter documents in Personal Document Store}
\begin{itemize}
    \item \textbf{Primary actor:} Registered Recipient
    \item \textbf{Interested parties:} 
        \begin{itemize}
            \item \textit{eDocs:} Wants to know popular searches for optimising document caching
        \end{itemize}

    \item \textbf{Preconditions:}
        \begin{itemize}
            \item User has consulted his personal document store (UC4).
        \end{itemize}

    \item \textbf{Postconditions:}
        \begin{itemize}
            \item The Registered Recipient has received a new view of documents in his Personal document store, matching the query he gave to the system.
        \end{itemize}
        
    \item \textbf{Main scenario:} 
    \begin{enumerate}
       \item The user indicates he wants to pose a filter query to the Personal Document Store
       \item The user receives a form with filter choices, which he completes, and indicates he wants to filter with given parameters.
       \item The System retrieves all documents of the Registered User matching his query.
       \item The System provides an overview of all retrieved documents matching the User's query.
    \end{enumerate}

    \item \textbf{Alternative scenarios:} 
    \begin{enumerate}
        \item [4a.] The System did not find any documents matching the User's query
        \item [5a.] The Registered User receives a notification that no documents have matched his filter parameters.
    \end{enumerate}
    
    \item \textbf{Remarks:}
        \begin{itemize}
            \item Abstraction is made of the contents of the filter form. The given choices can be free form or very simple ones.
        \end{itemize}
\end{itemize}

\subsection{\emph{UC8}: Consult Document Delivery status}
\begin{itemize}
    \item \textbf{Primary actor:} Customer Organisation Representative
    \item \textbf{Interested parties:} 
        \begin{itemize}
            \item \textit{eDocs:} wants to let the organisation know their documents are actually being delivered
        \end{itemize}

    \item \textbf{Preconditions:}
        \begin{itemize}
            \item The Representative is logged in (UC1).
        \end{itemize}

    \item \textbf{Postconditions:}
        \begin{itemize}
            \item Representative has received Delivery information regarding the documents he is interested in.
        \end{itemize}
        
    \item \textbf{Main scenario:} 
    \begin{enumerate}
       \item The Representative signals he wants an overview of the company documents to which he has access (see remarks).
       \item The System retrieves all documents managed by the Representative.
       \item The System presents the Representative with an overview of the matching documents, including a small individual note per document whether it has been delivered or not.
       \item The use case ends here.
    \end{enumerate}

    \item \textbf{Alternative scenarios:} 
    \begin{enumerate}
        \item [4a.] The Representative indicates he wants to see details (see remarks) of a document present in the overview.
        \item [5a.] The System presents the Representative with the detailed information for indicated document.
        \item [4b.] The Representative indicates he only wants to see undelivered documents.
        \item [5b.] The System retrieves all undelivered documents managed by the Representative.
        \item [6b.] The System presents the Representative with an overview of the retrieved documents.
    \end{enumerate}
    
    \item \textbf{Remarks:}
        \begin{itemize}
            \item A representative only has access to documents generated by his role in the company. E.g. Sales representatives cannot consult payslips, which are managed by Human Resources Representatives.
            \item A document in an online distribution platform is only flagged as delivered after the document has also been opened by the Recipient.
            \item Details of a document in this Use Case denote for example the chosen distribution channel, the contents of the document, the date of generation \ldots
        \end{itemize}
\end{itemize}

\subsection{\emph{UC9}: Submit Raw Data}
\begin{itemize}
    \item \textbf{Primary actor:} Customer Organisation or Social Secretary
    \item \textbf{Interested parties:} 
        \begin{itemize}
            \item \textit{Customer Organisation Representative:} wants to make sure the documents he manages are correctly generated and delivered from the raw data
            \item \textit{eDocs:} wants the system to correctly handle raw data or signal the company if errors exist
        \end{itemize}

    \item \textbf{Preconditions:}
        \begin{itemize}
            \item The primary actor is authenticated (UC1).
        \end{itemize}

    \item \textbf{Postconditions:}
        \begin{itemize}
            \item The System has received a correct batch of data which can be processed into docuents.
            \item The Customer Organisation knows that no errors are present in the currently submitted data.
        \end{itemize}
        
    \item \textbf{Main scenario:} 
    \begin{enumerate}
       \item The primary actor submits over agreed protocol a batch of raw data for processing.
       \item The System verifies that the raw data entries are all correctly formatted.
       \item The Use Case ends here.
    \end{enumerate}

    \item \textbf{Alternative scenarios:} 
    \begin{enumerate}
        \item [3a.] The System finds one or more entries in the raw data which are not correctly formatted.
        \item [4a.] The System collects all incorrect entries.
        \item [5a.] The System sends a notice to the responsible Customer Organisation Representative(s) that errors have been found in recently submitted data.
    \end{enumerate}
    
    \item \textbf{Remarks:}
        \begin{itemize}
            \item The Customer Organisation as specified in the primary actors can mean a specific user periodically submitting data batches, but is mostly implied to be an automated system in the company, handling the batches of raw data. Analogous for the Social Secretary.
            \item Authentication of the primary actor can be through an interactive login procedure, but could also be automated authentication for the company systems, so they can automatically and periodically submit accumulated batches of raw data without user intervention.
        \end{itemize}
\end{itemize}

\subsection{\emph{UC10}: Handle Raw Data error}
\begin{itemize}
    \item \textbf{Primary actor:} Customer Organisation Representative
    \item \textbf{Interested parties:} 
        \begin{itemize}
            \item \textit{Customer Organisation:} wants to make sure its documents are correctly generated and delivered
            \item \textit{Social Secretary:} wants know if errors existed in the data they submitted, to check if their bookkeeping is sound.
        \end{itemize}

    \item \textbf{Preconditions:}
        \begin{itemize}
            \item The System has received a batch of raw data and found errors in the entries
            \item The System has sent a notice to the responsible Customer Organisation Representative.
            \item The Customer Organisation Representative as primary actor is responsible for the document type for which an error was present in the raw data.
        \end{itemize}

    \item \textbf{Postconditions:}
        \begin{itemize}
            \item The System has received new raw data to replace the data entries which contained errors.
            \item No new errors are present in the newly submitted raw data entries.
        \end{itemize}
        
    \item \textbf{Main scenario:} 
    \begin{enumerate}
       \item The Representative receives a notice that errors were present in a recently submitted raw data batch.
       \item The Representative logs in to the system (UC1).
       \item The Representative indicates he wants to receive an overview of the data errors to be handled by him/her.
       \item The System collects all raw data errors for which the Representative is responsible.
       \item The System presents the Representative with an overview of retrieved errors.
       \item the Representative gathers the correct data.
       \item The Representative submits over agreed protocol the new batch of raw data for processing.
       \item The System checks the new raw data for errors.
       \item No errors are found and the system indicates to the Representative that his submission was succesful.
    \end{enumerate}

    \item \textbf{Alternative scenarios:} 
    \begin{enumerate}
        \item [9a.] Errors are detected in the newly submitted data.
        \item [10a.] The System sends a notice to the responsible Customer Organisation Representative(s) that errors have been found in recently submitted data.
    \end{enumerate}
    
    \item \textbf{Remarks:}
        \begin{itemize}
            \item Abstraction is made of the way the Representative receives the notification. This could be by email or an external notification system of the company or eDocs. In this case the Log-In step is necessary. If the notice is internal to the eDocs system, and the Representative only receives it after logging in, the Log-In (UC1) step becomes a precondition.
            \item The agreed protocol for submission by the Representative should include a way to signal that the submitted raw data replaces erronous data. It could be a more interactive way of submitting data, for example, a field or button in the overview of the retrieved errors.
        \end{itemize}
\end{itemize}

\subsection{\emph{UC11}: Name of use case 11}
\begin{itemize}
    \item \textbf{Primary actor:} primary actor
    \item \textbf{Interested parties:} 
        \begin{itemize}
            \item \textit{Name of interested party:} reason why party is interested
        \end{itemize}

    \item \textbf{Preconditions:}
        \begin{itemize}
            \item First precondition.
            \item Second precondition.
        \end{itemize}

    \item \textbf{Postconditions:}
        \begin{itemize}
            \item First postcondition.
            \item Second postcondition.
        \end{itemize}
        
    \item \textbf{Main scenario:} 
    \begin{enumerate}
       \item Step 1
       \item Step 2
       \item Step 3
       \item \ldots
    \end{enumerate}

    \item \textbf{Alternative scenarios:} 
    \begin{enumerate}
        \item [3b.] Alternative at step 3
    \end{enumerate}
    
    \item \textbf{Remarks:}
        \begin{itemize}
            \item First remark
        \end{itemize}
\end{itemize}

\subsection{\emph{UC12}: Name of use case 12}
\begin{itemize}
    \item \textbf{Primary actor:} primary actor
    \item \textbf{Interested parties:} 
        \begin{itemize}
            \item \textit{Name of interested party:} reason why party is interested
        \end{itemize}

    \item \textbf{Preconditions:}
        \begin{itemize}
            \item First precondition.
            \item Second precondition.
        \end{itemize}

    \item \textbf{Postconditions:}
        \begin{itemize}
            \item First postcondition.
            \item Second postcondition.
        \end{itemize}
        
    \item \textbf{Main scenario:} 
    \begin{enumerate}
       \item Step 1
       \item Step 2
       \item Step 3
       \item \ldots
    \end{enumerate}

    \item \textbf{Alternative scenarios:} 
    \begin{enumerate}
        \item [3b.] Alternative at step 3
    \end{enumerate}
    
    \item \textbf{Remarks:}
        \begin{itemize}
            \item First remark
        \end{itemize}
\end{itemize}

\subsection{\emph{UC13}: Name of use case 13}
\begin{itemize}
    \item \textbf{Primary actor:} primary actor
    \item \textbf{Interested parties:} 
        \begin{itemize}
            \item \textit{Name of interested party:} reason why party is interested
        \end{itemize}

    \item \textbf{Preconditions:}
        \begin{itemize}
            \item First precondition.
            \item Second precondition.
        \end{itemize}

    \item \textbf{Postconditions:}
        \begin{itemize}
            \item First postcondition.
            \item Second postcondition.
        \end{itemize}
        
    \item \textbf{Main scenario:} 
    \begin{enumerate}
       \item Step 1
       \item Step 2
       \item Step 3
       \item \ldots
    \end{enumerate}

    \item \textbf{Alternative scenarios:} 
    \begin{enumerate}
        \item [3b.] Alternative at step 3
    \end{enumerate}
    
    \item \textbf{Remarks:}
        \begin{itemize}
            \item First remark
        \end{itemize}
\end{itemize}

\subsection{\emph{UC14}: Name of use case 14}
\begin{itemize}
    \item \textbf{Primary actor:} primary actor
    \item \textbf{Interested parties:} 
        \begin{itemize}
            \item \textit{Name of interested party:} reason why party is interested
        \end{itemize}

    \item \textbf{Preconditions:}
        \begin{itemize}
            \item First precondition.
            \item Second precondition.
        \end{itemize}

    \item \textbf{Postconditions:}
        \begin{itemize}
            \item First postcondition.
            \item Second postcondition.
        \end{itemize}
        
    \item \textbf{Main scenario:} 
    \begin{enumerate}
       \item Step 1
       \item Step 2
       \item Step 3
       \item \ldots
    \end{enumerate}

    \item \textbf{Alternative scenarios:} 
    \begin{enumerate}
        \item [3b.] Alternative at step 3
    \end{enumerate}
    
    \item \textbf{Remarks:}
        \begin{itemize}
            \item First remark
        \end{itemize}
\end{itemize}

\subsection{\emph{UC15}: Name of use case 15}
\begin{itemize}
    \item \textbf{Primary actor:} primary actor
    \item \textbf{Interested parties:} 
        \begin{itemize}
            \item \textit{Name of interested party:} reason why party is interested
        \end{itemize}

    \item \textbf{Preconditions:}
        \begin{itemize}
            \item First precondition.
            \item Second precondition.
        \end{itemize}

    \item \textbf{Postconditions:}
        \begin{itemize}
            \item First postcondition.
            \item Second postcondition.
        \end{itemize}
        
    \item \textbf{Main scenario:} 
    \begin{enumerate}
       \item Step 1
       \item Step 2
       \item Step 3
       \item \ldots
    \end{enumerate}

    \item \textbf{Alternative scenarios:} 
    \begin{enumerate}
        \item [3b.] Alternative at step 3
    \end{enumerate}
    
    \item \textbf{Remarks:}
        \begin{itemize}
            \item First remark
        \end{itemize}
\end{itemize}

\subsection{\emph{UC16}: Name of use case 16}
\begin{itemize}
    \item \textbf{Primary actor:} primary actor
    \item \textbf{Interested parties:} 
        \begin{itemize}
            \item \textit{Name of interested party:} reason why party is interested
        \end{itemize}

    \item \textbf{Preconditions:}
        \begin{itemize}
            \item First precondition.
            \item Second precondition.
        \end{itemize}

    \item \textbf{Postconditions:}
        \begin{itemize}
            \item First postcondition.
            \item Second postcondition.
        \end{itemize}
        
    \item \textbf{Main scenario:} 
    \begin{enumerate}
       \item Step 1
       \item Step 2
       \item Step 3
       \item \ldots
    \end{enumerate}

    \item \textbf{Alternative scenarios:} 
    \begin{enumerate}
        \item [3b.] Alternative at step 3
    \end{enumerate}
    
    \item \textbf{Remarks:}
        \begin{itemize}
            \item First remark
        \end{itemize}
\end{itemize}

\section{Non-functional requirements}\label{sec:non-functional}
In this section, we model the non-functional requirements for the system in the
form of \emph{quality attribute scenarios}. We provide for each type
(availability, performance and modifiability) one requirement.

\subsection{Availability}
\subsubsection{\emph{Av1}: Name of the quality attribute scenario}
Shortly describe the context of the scenario.

\begin{itemize}
    \item \textbf{Source:} source
    \item \textbf{Stimulus:}
        \begin{itemize}
            \item Description of a first stimulus.
            \item Description of a second stimulus.
        \end{itemize}

    \item \textbf{Artifact:} the stimulated artifact
    \item \textbf{Environment:} the condition under which the stimulus occurs
    \item \textbf{Response:}
        \begin{itemize}
            \item Describe how the system should respond to the stimulus.
        \end{itemize}

    \item \textbf{Response measure:}
        \begin{itemize}
            \item Describe how the satisfaction of a response is measured.
        \end{itemize}
\end{itemize}

\subsection{Performance}
\subsubsection{\emph{P1}: Name of the quality attribute scenario}
Shortly describe the context of the scenario.

\begin{itemize}
    \item \textbf{Source:} source
    \item \textbf{Stimulus:}
        \begin{itemize}
            \item Description of a first stimulus.
            \item Description of a second stimulus.
        \end{itemize}

    \item \textbf{Artifact:} the stimulated artifact
    \item \textbf{Environment:} the condition under which the stimulus occurs
    \item \textbf{Response:}
        \begin{itemize}
            \item Describe how the system should respond to the stimulus.
        \end{itemize}

    \item \textbf{Response measure:}
        \begin{itemize}
            \item Describe how the satisfaction of a response is measured.
        \end{itemize}
\end{itemize}

\subsection{Modifiability}
\subsubsection{\emph{M1}: Name of the quality attribute scenario}
Shortly describe the context of the scenario.

\begin{itemize}
    \item \textbf{Source:} source
    \item \textbf{Stimulus:}
        \begin{itemize}
            \item Description of a first stimulus.
            \item Description of a second stimulus.
        \end{itemize}

    \item \textbf{Artifact:} the stimulated artifact
    \item \textbf{Environment:} the condition under which the stimulus occurs
    \item \textbf{Response:}
        \begin{itemize}
            \item Describe how the system should respond to the stimulus.
        \end{itemize}

    \item \textbf{Response measure:}
        \begin{itemize}
            \item Describe how the satisfaction of a response is measured.
        \end{itemize}
\end{itemize}

\end{document}
