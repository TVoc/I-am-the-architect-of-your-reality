\documentclass[a4paper,10pt]{article}

\usepackage[english]{babel}
\usepackage{graphicx}
\usepackage[colorlinks, linkcolor=black, citecolor=black, urlcolor=black]{hyperref}
\usepackage{geometry}
\usepackage{verbatim}
\geometry{tmargin=3cm, bmargin=2.2cm, lmargin=2.2cm, rmargin=2cm}
\usepackage{todonotes} %Used for the figure placeholders

% Your name and student number must be filled in on the title page found in
% titlepage.tex.

\begin{document}
\input{titlepage}

\tableofcontents
\newpage

\section{Domain analysis}\label{sec:domain}
\subsection{Domain models}
This section shows the domain model(s).

\begin{figure}[!htp]
    \centering
    \includegraphics[width=0.8\textwidth]{domain_model.png}
    %\missingfigure[figwidth=0.8\textwidth]{Domain model}
    \caption{The domain model for the system.}\label{fig:domain_model}
\end{figure}

\subsection{Domain constraints}
In this section we provide additional domain constraints.

\begin{itemize}
    \item A Document is only delivered to the Recipient it is addressed to
    \item A Template must have the fields that are mandatory for the document type it is a Template for
    \item A Payslip can only be sent to Employees the Customer Organization employs
    \item If the Customer Organizations outsources Payslip management to a Social Secretary, it cannot submit Payslip data itself
    \item While a Recipient is subscribed to a Personal Document Store, any document for which e-mail is specified as the Distribution Channel is instead delivered through the Personal Document Store
    \item A Recipient can have only one Personal Document Store
    \item While a Recipient is subscribed to an Online Distribution Platform that is also an external party with respect to eDocs (i.e. not the Personal Document Store), any documents of the type the Online Distribution Platform distributes is sent via the Online Distribution Platform regardless of the Distribution Platform the Customer Organization has selected
    \item If while constructing a Raw Data Entry no individual priority is explicitly selected, the individual priority is the same as the default processing priority of the Customer Organization
    \item The fields that are filled out in a Raw Data Entry must be a subset of the fields specified by its corresponding Template containing at least the mandatory fields
    \item An Invoice cannot be its own advance invoice
    \item An Invoice is always addressed to a Client
    \item A Payslip is always addressed to an Employee
\end{itemize}

\subsection{Glossary}
In this section, we provide a glossary of the most important terminology used
in this analysis.

\begin{itemize}
    \item \textbf{Customer Organization}: Those organizations that outsource their document processing to eDocs and are prepared to provide monetary compensation.
    \item \textbf{Data Supplier:} Those individuals who are a potential source of raw document data.
    \item \textbf{Customer Organization Representative}: Employees that are authorized by the Customer Organization that employs them to trigger document generation and delivery for certain types of documents according to their role in the Organization. Some examples are salespeople and human resources managers.
    \item \textbf{Social Secretary:} An organization to which management of an organization's payroll can be outsourced.
    \item \textbf{Template:} Provides the skeleton for a certain document type. It defines fields that can be filled out with the contents of a certain Raw Data Entry. It is possible that some fields are mandatory.
    \item \textbf{Raw data entry:} Contains raw data from which a document can be generated. It provides content that fills out its corresponding template.
    \item \textbf{Invoice:} A commercial document issued for the sale of products and/or services from a seller (i.e. the Customer Organization) to a buyer (i.e. the Client).
    \item \textbf{Payslip:} A document detailing the wage or salary the Customer Organization owes the Employee and how it is calculated.
    \item \textbf{Distribution Channel:} The medium through which a certain document is sent to its corresponding Recipient. Encompasses both off-line and on-line media.
    \item \textbf{Online Distribution Platform:} An online platform that distributes one or more types of documents to its registered clients. Encompasses both platforms that are internal to eDocs (e.g. the Personal Document Store) and platforms that are external to eDocs (e.g. Zoomit)
    \item \textbf{Personal Document Store:} A personal delivery channel that any Recipient can register for. Effectively replaces e-mail as Distribution Channel. Any documents that have been sent to a registered Recipient in the past can be consulted by the Recipient. 
\end{itemize}

\section{Functional requirements}\label{sec:functional}
\subsection*{Use case model}

\begin{figure}[!htp]
    \centering
    \includegraphics[width=\textwidth]{Use_case_diagram.png}
    %\missingfigure[figwidth=0.8\textwidth]{Use case model}
    \caption{Use case diagram for the system.}\label{fig:use_case_model}
\end{figure}

\subsection{\emph{UC3}: Deliver document via Personal Document Store}
\begin{itemize}
	\item \textbf{Primary actor:} Registered Recipient (see remark 2)
	\item \textbf{Interested parties:} 
	\begin{itemize}
		\item \textit{Customer Organization:} wants their documents to be delivered.
        \item \textit{Registered Recipient:} wants their documents delivered via the Personal Document Store, as requested.
	\end{itemize}
	
	\item \textbf{Preconditions:}
	\begin{itemize}
		\item A document has been generated that should be delivered via e-mail (as indicated by the Customer Organization).
	\end{itemize}
	
	\item \textbf{Postconditions:}
	\begin{itemize}
		\item A new document has been added to the Personal Document Store of the Registered Recipient.
		\item The System has sent a notification to the addressee of the document. This e-mail contains a short description of the document and a link to the document.
		\item The System has marked the corresponding job as added to Personal Document Store.
	\end{itemize}
	
	\item \textbf{Main scenario:} 
	\begin{enumerate}
		\item The System adds the document to the Personal Document Store of the Registered recipient.
		\item The System constructs an e-mail. This e-mail is addressed to the name of the Registered recipient and contains a short description of the received document (i.e. the sender of the document, the type of the document and the date at which the document was sent) and a link to the document.
		\item The System sends this e-mail to the e-mail address of the Registered recipient.
	\end{enumerate}
	
	\item \textbf{Alternative scenarios:} 
	None
	
	\item \textbf{Remarks:}
	\begin{itemize}
		\item This use case is an extension of UC15: Deliver document via e-mail
        \item The Recipient being a registered recipient implies the fact that they have a Personal Document Store, and documents marked to be delivered by e-mail should be instead delivered to this Personal Document Store.
	\end{itemize}
\end{itemize}

\subsection{\emph{UC7}: Filter documents in Personal Document Store}
\begin{itemize}
    \item \textbf{Primary actor:} Registered Recipient
    \item \textbf{Interested parties:} 
        \begin{itemize}
            \item \textit{eDocs:} wants to know popular searches for optimising document caching
            \item \textit{Registered Recipient:} wants to be able to find (a) certain document(s) more easily.
        \end{itemize}

    \item \textbf{Preconditions:}
        \begin{itemize}
            \item Registered Recipient has consulted his Personal Document Store (UC4).
        \end{itemize}

    \item \textbf{Postconditions:}
        \begin{itemize}
            \item The Registered Recipient has received a new view of documents in his Personal Document Store, matching the query he gave to the system.
        \end{itemize}
        
    \item \textbf{Main scenario:} 
    \begin{enumerate}
       \item The Registered Recipient indicates he wants to pose a filter query to the Personal Document Store
       \item The Registered Recipient receives a form with filter choices, which he completes, and indicates he wants to filter with given parameters.
       \item The System retrieves all documents of the Registered Recipient matching his query.
       \item The System provides an overview of all retrieved documents matching the Registered Recipient's query.
    \end{enumerate}

    \item \textbf{Alternative scenarios:} 
    \begin{enumerate}
        \item [4a.] The System did not find any documents matching the Registered Recipient's query
        \item [5a.] The Registered Recipient receives a notification that no documents have matched his filter parameters.
    \end{enumerate}
    
    \item \textbf{Remarks:}
        \begin{itemize}
            \item Abstraction is made of the contents of the filter form. The given choices can be free form or very simple ones.
        \end{itemize}
\end{itemize}

\subsection{\emph{UC8}: Consult Document Delivery status}
\begin{itemize}
    \item \textbf{Primary actor:} Customer Organisation Representative
    \item \textbf{Interested parties:} 
        \begin{itemize}
            \item \textit{eDocs:} wants to let the organisation know their documents are actually being delivered
            \item \textit{Customer Organisation:} wants to be able to check up on their document delivery
        \end{itemize}

    \item \textbf{Preconditions:}
        \begin{itemize}
            \item The Representative is logged in (UC1).
        \end{itemize}

    \item \textbf{Postconditions:}
        \begin{itemize}
            \item Representative has received delivery information regarding the documents he is interested in.
        \end{itemize}
        
    \item \textbf{Main scenario:} 
    \begin{enumerate}
       \item The Representative signals he wants an overview of the company documents to which he has access (see remarks).
       \item The System retrieves all documents managed by the Representative.
       \item The System presents the Representative with an overview of the matching documents, including a small individual note per document whether it has been delivered or not.
       \item The Use Case ends here.
    \end{enumerate}

    \item \textbf{Alternative scenarios:} 
    \begin{enumerate}
        \item [4a.] The Representative indicates he wants to see details (see remarks) of a document present in the overview.
        \item [5a.] The System presents the Representative with the detailed information for indicated document.
        \item [4b.] The Representative indicates he only wants to see undelivered documents.
        \item [5b.] The System retrieves all undelivered documents managed by the Representative.
        \item [6b.] The System presents the Representative with an overview of the retrieved documents.
    \end{enumerate}
    
    \item \textbf{Remarks:}
        \begin{itemize}
            \item A Representative only has access to documents generated by his role in the company. E.g. Sales Representatives cannot consult payslips, which are managed by Human Resources Representatives.
            \item A document in an online distribution platform is only flagged as delivered after the document has also been opened by the Recipient.
            \item Details of a document in this Use Case denote for example the chosen Distribution Channel, the contents of the document, the date of generation \ldots
        \end{itemize}
\end{itemize}

\subsection{\emph{UC9}: Submit Raw Data}
\begin{itemize}
    \item \textbf{Primary actor:} Customer Organisation or Social Secretary or Customer Organization Representative
    \item \textbf{Interested parties:} 
        \begin{itemize}
            \item \textit{Customer Organisation Representative:} wants to make sure the documents he manages are correctly generated and delivered from the raw data
            \item \textit{eDocs:} wants the system to correctly handle raw data or signal the company if errors exist
            \item \textit{Customer Organisation:} wants to outsource document processing, generation and delivery.
        \end{itemize}

    \item \textbf{Preconditions:}
        \begin{itemize}
            \item The primary actor is authenticated (UC1).
        \end{itemize}

    \item \textbf{Postconditions:}
        \begin{itemize}
            \item The System has received a correct batch of raw data which can be processed into docuents.
            \item The Customer Organisation knows that no errors are present in the currently submitted data.
        \end{itemize}
        
    \item \textbf{Main scenario:} 
    \begin{enumerate}
       \item The primary actor submits over agreed protocol a batch of raw data for processing.
       \item The System verifies that the raw data entries are all correctly formatted.
       \item The Use Case ends here.
    \end{enumerate}

    \item \textbf{Alternative scenarios:} 
    \begin{enumerate}
        \item [3a.] The System finds one or more entries in the raw data which are not correctly formatted.
        \item [4a.] The System collects all incorrect entries.
        \item [5a.] The System sends a notice to the responsible Customer Organisation Representative(s) that errors have been found in recently submitted data.
    \end{enumerate}
    
    \item \textbf{Remarks:}
        \begin{itemize}
            \item The Customer Organisation as specified in the primary actors can mean a specific user periodically submitting data batches, but is mostly implied to be an automated system in the company, handling the batches of raw data. Analogous for the Social Secretary.
            \item Authentication of the primary actor can be through an interactive login procedure, but could also be automated authentication for the company systems, so they can automatically and periodically submit accumulated batches of raw data without user intervention.
        \end{itemize}
\end{itemize}

\subsection{\emph{UC10}: Handle Raw Data error}
\begin{itemize}
    \item \textbf{Primary actor:} Customer Organisation Representative
    \item \textbf{Interested parties:} 
        \begin{itemize}
            \item \textit{Customer Organisation:} wants to make sure its documents are correctly generated and delivered
            \item \textit{Social Secretary:} wants know if errors existed in the data they submitted, to check if their bookkeeping is sound.
            \item \textit{Customer Organisation Representative:} wants the documents they are responsible for to be generated and handled without errors.
        \end{itemize}

    \item \textbf{Preconditions:}
        \begin{itemize}
            \item The System has received a batch of raw data and found errors in the entries
            \item The System has sent a notice to the responsible Customer Organisation Representative.
            \item The Customer Organisation Representative as primary actor is responsible for the document type for which an error was present in the raw data.
            \item The Representative has received a notice that errors were present in a recently submitted raw data batch.
            \item The Representative is logged in to the system (UC1).
        \end{itemize}

    \item \textbf{Postconditions:}
        \begin{itemize}
            \item The System has received new raw data to replace the data entries which contained errors.
            \item No new errors are present in the newly submitted raw data entries.
        \end{itemize}
        
    \item \textbf{Main scenario:} 
        \begin{enumerate}
           \item The Representative indicates he wants to receive an overview of the data errors to be handled by him/her.
           \item The System collects all raw data errors for which the Representative is responsible.
           \item The System presents the Representative with an overview of retrieved errors.
           \item the Representative gathers the correct data.
           \item The Representative submits over agreed protocol the new batch of raw data for processing.
           \item The System checks the new raw data for errors.
           \item No errors are found and the system indicates to the Representative that his submission was succesful.
        \end{enumerate}

    \item \textbf{Alternative scenarios:} 
        \begin{enumerate}
            \item [9a.] Errors are detected in the newly submitted data.
            \item [10a.] The System sends a notice to the responsible Customer Organisation Representative(s) that errors have been found in recently submitted data.
        \end{enumerate}
    
    \item \textbf{Remarks:}
        \begin{itemize}
            \item Abstraction is made of the way the Representative receives the notification. This could be by e-mail or an external notification system of the company or eDocs. In this case the Log-In step is necessary. If the notice is internal to the eDocs system, and the Representative only receives it after logging in, the Log-In (UC1) step becomes a precondition.
            \item The agreed protocol for submission by the Representative should include a way to signal that the submitted raw data replaces erronous data. It could be a more interactive way of submitting data, for example, a field or button in the overview of the retrieved errors.
        \end{itemize}
\end{itemize}

\subsection{\emph{UC11}: Handle E-mail delivery error}
\begin{itemize}
    \item \textbf{Primary actor:} Customer Organisation Representative
    \item \textbf{Interested parties:} 
        \begin{itemize}
            \item \textit{Customer Organisation:} wants their documents delivered correctly, and to have the correct Recipient Details.
            \item \textit{Customer Organisation Representative:} wants to know when a document he manages is not delivered correctly.
            \item \textit{Recipient:} wants to receive their documents.
        \end{itemize}

    \item \textbf{Preconditions:}
        \begin{itemize}
            \item A document has been sent to the Recipient via e-mail, but was not delivered.
            \item The Recipient's e-mail address in the Customer Organisation's database is wrong and not an existing e-mail address.
            \item The External E-mail Provider has sent a notice to the System that certain documents could not reach their destination.
            \item The System has notified the Representative of the fact that certain documents under his responsibility could not reach their destination.
            \item The Representative has received a notice that certain documents under his responsibility could not reach their destination.
            \item The Representative has logged in to the system (UC1).
        \end{itemize}

    \item \textbf{Postconditions:}
        \begin{itemize}
            \item The document has been sent to the Recipient via e-mail, and has been delivered.
            \item The Recipient's e-mail address in the Customer Organisation's database is correct.
        \end{itemize}
        
    \item \textbf{Main scenario:} 
    \begin{enumerate}
       \item The Representative indicates he wants to view the list of documents that could not reach their destination by e-mail.
       \item The System collects all documents which could not be delivered by e-mail for which the Representative is responsible.
       \item The System presents the Representative with an overview of the collected documents and the delivery details.
       \item The Representative indicates a document and supplies a corrected e-mail address.
       \item The System resends the document with the newly supplied e-mail address.
       \item The Representative indicates he is done supplying correct e-mail addresses.
       \item The Use Case ends here.
    \end{enumerate}

    \item \textbf{Alternative scenarios:} 
    \begin{enumerate}
        \item [8a.] The Representative indicates he wants to supply a correct e-mail address to another document.
        \item [9a.] The Use Case resumes from step 4.
    \end{enumerate}
    
    \item \textbf{Remarks:}
        \begin{itemize}
            \item Abstraction is made of the way the Representative receives the notification. This could be by e-mail or an external notification system of the company or eDocs. In this case the log-in step is necessary. If the notice is internal to the eDocs system, and the Representative only receives it after logging in, the log-in (UC1) step becomes a precondition.
            \item The gathering of correct e-mail addresses is implied as human interaction.
            \item The fact that the Customer Organisation database is updated with the correct e-mail address is implied as human interaction.
        \end{itemize}
\end{itemize}

\subsection{\emph{UC12}: Unregister from Personal Document Store}
\begin{itemize}
    \item \textbf{Primary actor:} Registered Recipient
    \item \textbf{Interested parties:} 
        \begin{itemize}
            \item \textit{eDocs:} wants to manage the Registered Recipients and not offer services to Recipients against their will.
            \item \textit{Registered Recipient:} wants to use another distribution channel
        \end{itemize}

    \item \textbf{Preconditions:}
        \begin{itemize}
            \item The Recipient is a Registered Recipient.
            \item The Recipient is logged in (UC1).
        \end{itemize}

    \item \textbf{Postconditions:}
        \begin{itemize}
            \item The Recipient is now an Unregistered Recipient.
            \item The details of the Recipient are removed from the System.
            \item The Recipient is logged out.
        \end{itemize}
        
    \item \textbf{Main scenario:} 
    \begin{enumerate}
       \item The Recipient indicates he wants to unregister from the Personal Document Store.
       \item The System prompts him if he wants to continue with this action.
       \item The Recipient indicates that he wants to unregister.
       \item The System removes the Recipients details internally.
       \item The System marks the Recipient as unregistered.
       \item The System notifies the Recipient that he has been unregistered.
       \item The System logs out the Recipient.
    \end{enumerate}

    \item \textbf{Alternative scenarios:} 
    \begin{enumerate}
        \item [3a.] The Recipient indicates he does not want to unregister.
        \item [4a.] The Use Case ends here.
    \end{enumerate}
    
    \item \textbf{Remarks:}
        \begin{itemize}
            \item Documents in the Personal Document Store are not affected. The documents are present in the Document Archive, and a Personal Document Store can be reconstructed for a Recipient from this Archive.
        \end{itemize}
\end{itemize}

\subsection{\emph{UC13}: Deliver generated document}
\begin{itemize}
    \item \textbf{Primary actor:} Recipient
    \item \textbf{Interested parties:} 
        \begin{itemize}
            \item \textit{eDocs:} wants documents to be delivered with as little human interaction as possible
            \item \textit{Customer Organization:} wants their documents to be delivered
            \item \textit{Customer Organization Representative:} want to have a clear and detailed overview of the status of their document processing jobs
            \item \textit{Recipient:} wants to receive the documents addressed to them, with the requested delivery method.
        \end{itemize}

    \item \textbf{Preconditions:}
        \begin{itemize}
            \item Either the Customer Organisation or a Customer Organisation Representative has delivered a Data Batch containing the information necessary to generate the document.
            \item The document has been generated successfully.
        \end{itemize}

    \item \textbf{Postconditions:}
        \begin{itemize}
            \item The document has been sent on its way via the chosen Distribution Channel
            \item If this document is not part of a recurring batch of document processing jobs, the System has billed the customer organisation for it.
        \end{itemize}
        
    \item \textbf{Main scenario:} 
    \begin{enumerate}
       \item The System selects a Distribution Channel specified by the raw data or according to information the System knows about the Recipient (see remark 1)
       \item If the document was not a part of an automatic Data Batch, the Customer Organisation is billed with the costs of sending the document
       \item The System marks the document as sent.
    \end{enumerate}

    \item \textbf{Alternative scenarios:} 
    None
    
    \item \textbf{Remarks:}
        \begin{itemize}
            \item Step 1 is an extension point for other use cases
        \end{itemize}
\end{itemize}

\subsection{\emph{UC14}: Deliver document via print \& postal service}
\begin{itemize}
    \item \textbf{Primary actor:} Recipient
    \item \textbf{Interested parties:} 
        \begin{itemize}
            \item \textit{Print \& postal service:} provides delivery of traditional mail
        \end{itemize}

    \item \textbf{Preconditions:}
        \begin{itemize}
            \item The document's associated raw data specifies print \& postal service as the delivery method
        \end{itemize}

    \item \textbf{Postconditions:}
        \begin{itemize}
            \item The System has sent a PDF representing the document to the print \& postal service via a web service.
        \end{itemize}
        
    \item \textbf{Main scenario:} 
    \begin{enumerate}
       \item The System retrieves the Recipient's postal address from the raw data.
       \item The System sends the PDF representing the document to the print \& postal service via a web service, along with the Recipient's postal address.
    \end{enumerate}

    \item \textbf{Alternative scenarios:} 
    None
    
    \item \textbf{Remarks:}
        \begin{itemize}
            \item This use case is an extension of UC13: Deliver generated document
        \end{itemize}
\end{itemize}

\subsection{\emph{UC15}: Deliver document via e-mail}
\begin{itemize}
    \item \textbf{Primary actor:} Unregistered Recipient (see remark 3)
    \item \textbf{Interested parties:} 
        \begin{itemize}
            \item \textit{E-mail Provider:} Is responsible for delivering e-mail addressed to its clients
        \end{itemize}

    \item \textbf{Preconditions:}
        \begin{itemize}
            \item The document's associated raw data specifies e-mail as the delivery method
        \end{itemize}

    \item \textbf{Postconditions:}
        \begin{itemize}
            \item The System has sent an e-mail to the Recipient with PDF representing the document included as an attachment
        \end{itemize}
        
    \item \textbf{Main scenario:} 
    \begin{enumerate}
       \item The System retrieves the Recipient's e-mail address from the document's raw data.
       \item The System composes an e-mail and includes the PDF representing the document as an attachment
       \item The System sends the e-mail on its way
    \end{enumerate}

    \item \textbf{Alternative scenarios:} 
    \begin{enumerate}
        \item [2a.] If receipt tracking is enabled, the System composes an e-mail consisting of a short description of the document and a unique URL providing access to the document.
        \item [3a.] The System sends the e-mail on its way
        \item [4a.] The System arranges for an event to be fired that marks delivery failure should the Recipient not have followed the link within 30 days
    \end{enumerate}
    
    \item \textbf{Remarks:}
        \begin{itemize}
            \item This use case is an extension of UC13: Deliver generated document.
            \item Step 2 is an extension point for other use cases.
            \item The Recipient being Unregistered implies he is not the owner of a Personal document store.
        \end{itemize}
\end{itemize}

\subsection{\emph{UC16}: Deliver document via Online Distribution Platform}
\begin{itemize}
    \item \textbf{Primary actor:} Recipient
    \item \textbf{Interested parties:} 
        \begin{itemize}
            \item \textit{Online Distribution Platform:} wants to deliver documents addressed to its clients
        \end{itemize}

    \item \textbf{Preconditions:}
        \begin{itemize}
            \item The document's associated raw data specifies a certain Online Distribution Platform as the delivery method, accompanied by the necessary details for the Platform
        \end{itemize}

    \item \textbf{Postconditions:}
        \begin{itemize}
            \item The System has sent the PDF representing the document to the Online Distribution Platform via a web service along with information that identifies the Recipient
        \end{itemize}
        
    \item \textbf{Main scenario:} 
    \begin{enumerate}
       \item The System retrieves information that identifies the Recipient from the document's raw data
       \item The System sends the PDF representing the information and the identifying information via a web service to the Online Distribution Platform
       \item If the Customer Organization has requested receipt tracking and if the Online Distribution Platform supports it, the System indicates this to the Online Distribution Platform
    \end{enumerate}

    \item \textbf{Alternative scenarios:} 
    None
    
    \item \textbf{Remarks:}
        \begin{itemize}
          \item This use case is an extension of UC13: Deliver generated document
        \end{itemize}
\end{itemize}

\subsection{\emph{UC17}: Consult Billing Information}
\begin{itemize}
	\item \textbf{Primary actor:} Customer Organization
	\item \textbf{Interested parties:} 
	\begin{itemize}
		\item \textit{eDocs:} wants to store all information concerning financial transactions
		\item \textit{Social Secretary:} wants to know the costs it has incurred on behalf of the Customer Organization
	\end{itemize}
	
	\item \textbf{Preconditions:}
	\begin{itemize}
		\item The user is logged in (UC1: Log in)
	\end{itemize}
	
	\item \textbf{Postconditions:}
	\begin{itemize}
		\item The System has presented the User with an overview of billing information
	\end{itemize}
	
	\item \textbf{Main scenario:} 
	\begin{enumerate}
		\item The User indicates he wishes to view the Customer Organization's billing information
		\item The System collects the Customer Organization's billing information
		\item The System presents the user with an overview of the retrieved billing information
		\item The Use Case ends
	\end{enumerate}
	
	\item \textbf{Alternative scenarios:} 
	\begin{enumerate}
		\item [4a.] The User indicates that he wants a more detailed view of a specific transaction
		\item [5a.] The System presents the User with a more detailed view of that transaction
		\item [6a.] The User indicates he wants to return to the overview
		\item [7a.] The Use Case goes back to step 3
	\end{enumerate}
	
	\item \textbf{Remarks:}
	   \begin{itemize}
          \item The Billing Information in this Use Case pertains the cost of the Customer Organisation for using the eDocs system. E.g. monthly fees or priority costs.
          \item The ``User'' is this Use Case is a Customer Organisation Employee or Representative who is responsible for the contractual interaction with the eDocs company.
        \end{itemize}
\end{itemize}

\subsection{\emph{UC18}: Manage Customer Organization information}
\begin{itemize}
	\item \textbf{Primary actor:} eDocs System Administrator
	\item \textbf{Interested parties:} 
	\begin{itemize}
		\item \textit{Customer Organization:} desires an adequate level of service from eDocs
	\end{itemize}
	
	\item \textbf{Preconditions:}
	\begin{itemize}
		\item The Customer Organization has been registered with the System
		\item The eDocs System Administrator is logged in (UC1: Log in)
	\end{itemize}
	
	\item \textbf{Postconditions:}
	\begin{itemize}
		\item Any changes made by the eDocs System Administrator have been logged
	\end{itemize}
	
	\item \textbf{Main scenario:} 
	\begin{enumerate}
		\item The Administrator indicates he wants an overview of all Customer Organizations
		\item The System collects the information and provides the requested overview
		\item The Administrator indicates he wants to edit a certain Customer Organization's information (e.g. the Service Level Agreement has been revised and any changes need to be logged)
		\item The System provides the Administrator with an overview of the Customer Organization's information and provides the means to edit it
		\item The Administrator offers to the System the information he wants to change
		\item The System processes the changes and finds no problems; the Administrator is notified of success
		\item The Representative indicates he has finished editing Customer Organization information
		\item The use case ends
	\end{enumerate}
	
	\item \textbf{Alternative scenarios:} 
	\begin{enumerate}
		\item [6a.] During processing of the changes, the System finds a problem.
		\item [7a.] The System indicates to the Administrator that the new Details are not valid and not updated internally.
		\item [8a.] The Use Case resumes from step 4.
		\item [7b.] The Representative indicates he wants to update the Details of another Recipient.
		\item [8b.] The Use Case resumes from step 2.
		\item [2c.] The Administrator indicates he is done managing Customer Organization Details.
		\item [3c.] The Use Case ends here.
		\item [4d.] The Representative indicates he does not want to update the presented Recipient's details.
		\item [5d.] The Use Case resumes from step 2.
	\end{enumerate}
	
	\item \textbf{Remarks:}
	None
\end{itemize}

\subsection{\emph{UC19}: Register Customer Organization}
\begin{itemize}
	\item \textbf{Primary actor:} eDocs System Administrator
	\item \textbf{Interested parties:} 
	\begin{itemize}
		\item \textit{eDocs:} wants as many clients to use their services as possible
		\item \textit{Customer Organization:} wants to use eDocs' services and is prepared to provide monetary compensation
	\end{itemize}
	
	\item \textbf{Preconditions:}
	\begin{itemize}
		\item Negotiations regarding the Service Level Agreement were successful
		\item The eDocs System Administrator is logged in (UC1: Log in)
	\end{itemize}
	
	\item \textbf{Postconditions:}
	\begin{itemize}
		\item An account for the Customer Organization has been opened in addition to accounts for any representatives, if applicable
	\end{itemize}
	
	\item \textbf{Main scenario:} 
	\begin{enumerate}
		\item The eDocs System Administrator indicates he wants to register a new Customer Organization
		\item The System provides the Administrator with the means to input the required information (e.g. default priority level, credentials for any automatic data delivery channels, etc.)
		\item The Administrator inputs the required information
		\item The Administrator indicates he has finished providing the information
		\item The System creates an account on behalf of the Customer Organization and any representatives, if applicable
		\item The use case ends
	\end{enumerate}
	
	\item \textbf{Alternative scenarios:} 
	\begin{enumerate}
		\item [5a.] The System indicates that the information provided was incomplete or invalid
		\item [6a.] The use case goes back to step 3
	\end{enumerate}
	
	\item \textbf{Remarks:}
	None
\end{itemize}

\subsection{\emph{UC20}: Unregister Customer Organization}
\begin{itemize}
	\item \textbf{Primary actor:} eDocs System Administrator
	\item \textbf{Interested parties:} 
	\begin{itemize}
		\item \textit{eDocs:} wishes to have an accurate view of their client base
		\item \textit{Customer Organization:} no longer wants to be serviced by eDocs
	\end{itemize}
	
	\item \textbf{Preconditions:}
	\begin{itemize}
		\item The Customer Organization has been registered
		\item The Customer Organization has authorized account removal
		\item The eDocs System Administrator is logged in (UC1: Log in)
		\item The eDocs System Administrator has consulted the Customer Organization's information (UC18: Manage Customer Organization Information)
	\end{itemize}
	
	\item \textbf{Postconditions:}
	\begin{itemize}
		\item All accounts associated with the Customer Organization have been removed. Any documents generated on the Customer Organization's behalf remain untouched.
	\end{itemize}
	
	\item \textbf{Main scenario:} 
	\begin{enumerate}
		\item The eDocs System Administrator indicates he wishes to unregister the Customer Organization
		\item The System removes all accounts associated with the Customer Organization
	\end{enumerate}
	
	\item \textbf{Alternative scenarios:} 
	None
	
	\item \textbf{Remarks:}
	None
\end{itemize}

\subsection{\emph{UC21}: Confirm delivery through receipt tracking}
\begin{itemize}
	\item \textbf{Primary actor:} Recipient
	\item \textbf{Interested parties:} 
	\begin{itemize}
		\item \textit{eDocs:} wishes to assure the Company Organization that their documents are being delivered
		\item \textit{Customer Organization:} wishes to be certain that documents sent on their behalf have been delivered
	\end{itemize}
	
	\item \textbf{Preconditions:}
	\begin{itemize}
		\item Receipt tracking was activated for the delivered document
	\end{itemize}
	
	\item \textbf{Postconditions:}
	\begin{itemize}
		\item The System has marked the document as delivered
	\end{itemize}
	
	\item \textbf{Main scenario:} 
	\begin{enumerate}
		\item The Recipient triggers receipt confirmation (see remark 1)
		\item The System marks the relevant document as delivered
	\end{enumerate}
	
	\item \textbf{Alternative scenarios:} 
	None
	
	\item \textbf{Remarks:}
	\begin{itemize}
		\item Step 1 is an extension point for other use cases
	\end{itemize}
\end{itemize}

\subsection{\emph{UC22}: Confirm delivery through e-mail receipt tracking}
\begin{itemize}
	\item \textbf{Primary actor:} Recipient
	\item \textbf{Interested parties:} 
	\begin{itemize}
		\item \textit{eDocs:} wishes to assure the Company Organization that their documents are being delivered
		\item \textit{Customer Organization:} wishes to be certain that documents sent on their behalf have been delivered
	\end{itemize}
	
	\item \textbf{Preconditions:}
	\begin{itemize}
		\item The document was sent by e-mail
	\end{itemize}
	
	\item \textbf{Postconditions:}
	\begin{itemize}
		\item The System has marked the document as delivered
	\end{itemize}
	
	\item \textbf{Main scenario:} 
	\begin{enumerate}
		\item The Recipient follows the unique URL provided in the e-mail
		\item The System registers that the Recipient has followed the URL
	\end{enumerate}
	
	\item \textbf{Alternative scenarios:} 
	\begin{enumerate}
		\item [2a.] The System has previously perceived that the Recipient did not follow the URL within 30 days of the System sending the document and therefore does not respond to the Recipient following the link beyond indicating that the URL is no longer valid
		\item [3a.] The use case ends
	\end{enumerate}
	
	\item \textbf{Remarks:}
	\begin{itemize}
		\item This use case is an extension of UC23: Confirm delivery through receipt tracking
	\end{itemize}
\end{itemize}

\subsection{\emph{UC23}: Confirm delivery through Online Distribution Platform receipt tracking}
\begin{itemize}
	\item \textbf{Primary actor:} Online Distribution Platform
	\item \textbf{Interested parties:} 
	\begin{itemize}
		\item \textit{eDocs:} wishes to assure the Company Organization that their documents are being delivered
		\item \textit{Customer Organization:} wishes to be certain that documents sent on their behalf have been delivered
        \item \textit{Online Distribution Platform:} wants to signal accurately to the sender when a documents has been delivered.
	\end{itemize}
	
	\item \textbf{Preconditions:}
	\begin{itemize}
		\item Receipt tracking has been requested by the Customer Organization
		\item The Online Distribution Platform supports receipt tracking
	\end{itemize}
	
	\item \textbf{Postconditions:}
	\begin{itemize}
		\item The System has marked the document as delivered
	\end{itemize}
	
	\item \textbf{Main scenario:} 
	\begin{enumerate}
		\item The Online Distribution platform indicates that the Recipient has received the document
		\item The System identifies the document that has been received
	\end{enumerate}
	
	\item \textbf{Alternative scenarios:} 
	\begin{enumerate}
		\item [2a.] The Online Distribution Platform does not supply sufficient information to identify the document that has been delivered
		\item [3a.] The System sends a notification to the Online Distribution Platform and logs the anomaly for review by an eDocs System Administrator
	\end{enumerate}
	
	\item \textbf{Remarks:}
	\begin{itemize}
		\item This use case is an extension of UC23: Confirm delivery through receipt tracking
	\end{itemize}
\end{itemize}

\subsection{\emph{UC24}: Submit document template}
\begin{itemize}
	\item \textbf{Primary actor:} Customer Organization Representative
	\item \textbf{Interested parties:} 
	\begin{itemize}
		\item \textit{eDocs:} wants to simplify document generation and delivery
		\item \textit{Customer Organization:} wants to outsource document generation and delivery
	\end{itemize}
	
	\item \textbf{Preconditions:}
	\begin{itemize}
		\item The Customer Organization Representative is logged in (UC1: Log in)
	\end{itemize}
	
	\item \textbf{Postconditions:}
	\begin{itemize}
		\item The new template will be used for the specified document type
	\end{itemize}
	
	\item \textbf{Main scenario:} 
	\begin{enumerate}
		\item The Representative indicates he wants to upload a new template for a certain document type he is authorized to generate
		\item The System provides the Representative with the means to upload a new template
		\item The Representative uploads the new template
		\item The System verifies the validity of the template
		\item The System logs that the new template will be used for the specified document type. In case a template for the document type already existed, it is overwritten
	\end{enumerate}
	
	\item \textbf{Alternative scenarios:} 
	\begin{enumerate}
		\item [5a.] The System has determined that the template has been formatted incorrectly or is incomplete (e.g. the document type is invoice and there is no net amount field) and shows the Representative which errors he has made
		\item [6a.] The use case goes back to step 2
	\end{enumerate}
	
	\item \textbf{Remarks:}
	None
\end{itemize}


\section{Non-functional requirements}\label{sec:non-functional}
In this section, we model the non-functional requirements for the system in the
form of \emph{quality attribute scenarios}. We provide for each type
(availability, performance and modifiability) one requirement.

\subsection{Availability}
\subsubsection{\emph{Av1}: Submission system failure}
The subsystem accepting connections for raw data submission fails or crashed. External connections are refused and Customer Organisations and Social Secretaries are unable to submit new raw data batches.

\begin{itemize}
    \item \textbf{Source:} Internal
    \item \textbf{Stimulus:}
        \begin{itemize}
            \item The subsystem for accepting raw data batches fails or crashes
        \end{itemize}

    \item \textbf{Artifact:} Internal subsystem and external communication channel(s)
    \item \textbf{Environment:} Normal execution
    \item \textbf{Response:}
        \begin{itemize}
            \item Raw data which is still passing through the subsystem is not lost when the failure occurs.
            \item This does not affect the generation of documents for which the raw data was already passed to the rest of the system by the submission subsystem.
            \item Prevention:
            \begin{itemize}
                \item The submission system should have a guaranteed minimal up-time.
            \end{itemize}
            \item Detection:
            \begin{itemize}
                \item The eDocs Operators are notified of this problem.
                \item The System is able to detect this problem and goes into degraded mode:
                \begin{itemize}
                    \item Raw data which passes through the subsystem is always saved to secondary (persistent) storage before an acknowledgement to the submitting client is sent. This way, no raw data is acknowledged but lost when the system fails. This data is deleted from the secondary storage after being passed to the rest of the system.
                    \item Fail gracefully: when a client tries to submit data while the subsystem has failed, he receives an alternate response over the submission protocol, stating that the submission system is temporarily unavailable. This response is given by a backup networking module, that takes over the external communication channel(s) when the submission subsystem fails.
                \end{itemize}
            \end{itemize}
            \item Resolution:
            \begin{itemize}
                \item The eDocs Operators verify the integrity of the secondary storage and otherwise notify all clients that raw data might have been lost, and they should make contact to verify their data and documents.
                \item The eDocs Operators replace or restart failed components, be it hardware or software.
                \item The subsystem then resumes by processing the remaining data in the secondary storage and again accepting external connections.
            \end{itemize}
        \end{itemize}

    \item \textbf{Response measure:}
        \begin{itemize}
            \item Prevention
            \begin{itemize}
                \item The subsystem should accept raw data for at least 99.5 percent of the time.
                \item Raw data can be corrupted or lost in only up to 5 percent of failure cases.
            \end{itemize}
            \item Detection
            \begin{itemize}
                \item The secondary storage should store at least the total amount of recurring data batches, as agreed in each client's SLA.
                \item Detection of the failure should happen within 30 seconds.
                \item It should take the backup network module maximally 5 seconds to take over the external communication channels.
            \end{itemize}
            \item Resolution
            \begin{itemize}
                \item Verifying the secondary storage should take up to 1 minute per client.
                \item The total problem addressing time of the eDocs Operators should be under 45 minutes or as many minutes as there are clients, whichever is the greatest.
            \end{itemize}
        \end{itemize}
\end{itemize}

\subsection{Performance}
\subsubsection{\emph{P1}: Generation of documents}
The System should be able to keep up with data submissions from all clients, at the very least in the situation where the amount of submitted data is the total of the standard periodic submissions as indicated in all individual SLA's. The System should also be able to honor the priority settings on the raw data batches as best as possible.

\begin{itemize}
    \item \textbf{Source:} Customer Organisation(s) and/or Social Secretarie(s) and/or Customer Organisation Representative(s)
    \item \textbf{Stimulus:}
        \begin{itemize}
            \item The source submits a batch of raw data.
            \item The submission subsystem starts and processes raw data from the secondary storage.
        \end{itemize}

    \item \textbf{Artifact:} The subsystem responsible for compiling raw data into documents and scheduling the generation relative to raw data priorities.
    \item \textbf{Environment:} Normal Execution
    \item \textbf{Response:}
        \begin{itemize}
            \item The System should be provided with enough processing power to process at least the mean load of document generation as stipulated by the SLA's. When the load on the document generation is not over the capacity of the system, this is viewed as ``normal circumstances''.
            \item Under normal circumstances, no extreme care should be taken with document priorities. They should be honored, and only lightly enforced, in case the system comes under heavy load at a near point in the future.
            \item When the load on the system becomes too high, and raw data is not immediately processed anymore, priorities should be strictly enforced as deadlines, with earliest deadline scheduling. The deadline for a document's generation is its priority maximum handling time (see description), added to its submission time.
            \item When the system comes under too heavy a load and priority honoring can not be guaranteed anymore, strict priority behaviour is still in effect:
            \begin{itemize}
                \item Of all conflicting/unreachable deadlines, documents with the Critical priority are the first to be handled.
                \item Next most important in order are Diamond, Gold and Silver. These are each respectively more important than the next. This means that Silver priority documents are sacrificed for the other priorities when necessary, then Gold priority documents if necessary, etc.
            \end{itemize}
        \end{itemize}

    \item \textbf{Response measure:}
        The different document priority classes have different guarantees of priority honoring, apart from their maximum generation time:
        \begin{itemize}
            \item For Silver priority documents, under heavy load, 90 percent of the documents should be generated within set deadline.
            \item For Gold priority documents, under heavy load, 97.5 percent of the documents should be generated within set deadline.
            \item For Diamond priority documents, under heavy load, 99 percent of the documents should be generated within set deadline.
            \item For Critical priority documents, under heavy load, all of the documents should be generated within set deadline.
        \end{itemize}
        Mainly due to this last guarantee, the eDocs administrators should be able to acquire extra processing power to comply with these guarantees:
        \begin{itemize}
            \item The eDocs administrators should receive a notice within 5 seconds when the system has come under heavy load.
            \item The eDocs administrators should receive a notice when the system has been under heavy load for a minute.
            \item The eDocs administrators should receive an urgent notice when the system has been under heavy load for 15 minutes.
            \item The eDocs administrators should be able to acquire external processing power within 1.5 hours of receiving this urgent notice.
        \end{itemize}
\end{itemize}

\subsection{Modifiability}
\subsubsection{\emph{M1}: New Document Type}
Due to multiple requests from different clients or based on market research, a new Document Type, e.g. a newsletter should be offered by the system, as an economic strategy. This new document type should not add any new steps to the data submission procedure, and should target a specific type of Recipient.

\begin{itemize}
    \item \textbf{Source:} End user or the finance department
    \item \textbf{Stimulus:} One of the following occurs
        \begin{itemize}
            \item End users wish to be able to outsource document processing for a new document type
            \item It is predicted that being able to process a new document type would generate a lot of business
        \end{itemize}

    \item \textbf{Artifact:} The Raw Data Submission sub-system and Document Generation sub-system
    \item \textbf{Environment:} Run-time or design time
    \item \textbf{Response:}
        \begin{itemize}
            \item The developer identifies what a document of the new document type should generally look like, taking into account mandatory fields and lay-out imposed by any legal constraints that apply.
            \item The developer identifies which identifying information must be supplied to an Online Distribution Platform in case it is chosen as the Distribution Channel for a document of the new type.
            \item The developer makes fitting changes to the Document Generation sub-system and the Raw Data Submission sub-system in order to support the new document type.
        \end{itemize}

    \item \textbf{Response measure:}
        \begin{itemize}
            \item Providing support for a new document type should take no more than one man week.
        \end{itemize}
\end{itemize}

\end{document}
