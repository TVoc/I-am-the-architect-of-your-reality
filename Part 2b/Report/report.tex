\documentclass[a4paper,10pt]{article}

\usepackage[english]{babel}
\usepackage{graphicx}
\usepackage[colorlinks, allcolors=black]{hyperref}
\usepackage{geometry}
\geometry{tmargin=3cm, bmargin=2.2cm, lmargin=2.2cm, rmargin=2cm}
\usepackage{todonotes} %Used for the figure placeholders
\usepackage[space]{grffile}

% Your name and student number must be filled in on the title page found in
% titlepage.tex.

\begin{document}
\input{titlepage}

\tableofcontents
\newpage

\section{Introduction}\label{sec:introduction}

The goal of this project was to design an architecture for a Document Processing system. Companies can use this system to submit data for payslips and invoices, and the corresponding documents are generated and delivered for them. This way, they do not have to provide the infrastructure for this themselves. The architecture as developed and explained in the following sections conforms to the functional and non-functional requirements as imposed by the assignment.

\section{Overview}\label{sec:overview}
\subsection{Architectural decisions}

\paragraph{Av1a\@: notification of the eDocs Administrator} Provide a brief discussion of the
decisions related to \emph{ReqX}.\\
\emph{Employed tactics and patterns: List all patterns and tactics used to
    achieve ReqX, if any.}
    
\paragraph{Av1b\@: storing the status of an individual job} Provide a brief discussion of the
decisions related to \emph{ReqX}.\\
\emph{Employed tactics and patterns: List all patterns and tactics used to
    achieve ReqX, if any.}
    
\paragraph{Av2a\@: notifying the appropriate operator within 1 minute} Provide a brief discussion of the decisions related to \emph{ReqX}.\\
\emph{Employed tactics and patterns: List all patterns and tactics used to
    achieve ReqX, if any.}
    
\paragraph{Av2b\@: temporary storage of Personal Document Store documents} Provide a brief discussion of the decisions related to \emph{ReqX}.\\
\emph{Employed tactics and patterns: List all patterns and tactics used to
    achieve ReqX, if any.}
    
\paragraph{Av3\@: Zoomit failure} Provide a brief discussion of the
decisions related to \emph{ReqX}.\\
\emph{Employed tactics and patterns: List all patterns and tactics used to
    achieve ReqX, if any.}
    
\paragraph{P2\@: document lookups} Provide a brief discussion of the
decisions related to \emph{ReqX}.\\
\emph{Employed tactics and patterns: List all patterns and tactics used to
    achieve ReqX, if any.}
    
\paragraph{P3\@: status overview for Customer Administrators} Provide a brief discussion of the
decisions related to \emph{ReqX}.\\
\emph{Employed tactics and patterns: List all patterns and tactics used to
    achieve ReqX, if any.}
    
\paragraph{M1\@: New type of document - bank statements} Provide a brief discussion of the
decisions related to \emph{ReqX}.\\
\emph{Employed tactics and patterns: List all patterns and tactics used to
    achieve ReqX, if any.}
    
\paragraph{M2\@: Multiple print & postal services} Provide a brief discussion of the
decisions related to \emph{ReqX}.\\
\emph{Employed tactics and patterns: List all patterns and tactics used to
    achieve ReqX, if any.}
    
\paragraph{M3\@: Dynamic selection of the cheapest of print & postal services} Provide a brief discussion of the decisions related to \emph{ReqX}.\\
\emph{Employed tactics and patterns: List all patterns and tactics used to
    achieve ReqX, if any.}

\subsection{Discussion}
Use this section to discuss your architecture in retrospect.
For example, what are the strong points of your architecture?
What are the weak points? Is there anything you would have done otherwise with
your current experience?
Are there any remarks about the architecture that you would give to your
customers?
Etc.

\section{Client-server view (UML Component diagram)}\label{sec:client-server}
The context diagram of the client-server view.
Discuss which components communicate with external components and what these
external components represent.

\begin{figure}[!htp]
    \centering
    %\includegraphics[width=\textwidth]{}
    \missingfigure[figwidth=0.8\textwidth]{Context diagram of the client-server
        view.}
    \caption{Context diagram for the client-server view.
        }\label{fig:cc-context}
\end{figure}

The primary diagram and accompanying explanation.

\begin{figure}[!htp]
    \centering
    %\includegraphics[width=\textwidth]{}
    \missingfigure[figwidth=0.8\textwidth]{Primary diagram of the client-server
        view.}
    \caption{Primary diagram of the client-server view.}\label{fig:cs-primary}
\end{figure}

\subsection{Main architectural decisions}
Discuss your architectural decisions for the most important requirements in
more detail using the components of the client-server view.
Pay attention to the solutions that you employed and the alternatives that you
considered.
The explanation here must be self-contained and complete.
Imagine you had to describe how the architecture supports the core
functionality to someone that is looking at the client-server view only.
Hide unnecessary details (these should be shown in the decomposition view).

\subsubsection{ReqX\@: requirement name}
Describe the design choices related to \emph{ReqX} together with the rationale
of why these choices where made.

\subsubsection*{Alternatives considered}
\paragraph{Alternative(s) for choice 1} Explain what alternative(s) you
considered for this design choice and why they where not selected.

\section{Decomposition view (UML Component diagram)}\label{sec:decomposition}
Discuss the decompositions of the components of the client-server view which
you have further decomposed.

\subsection{ComponentX}
\begin{figure}[!htp]
    \centering
    %\includegraphics[width=\textwidth]{}
    \missingfigure[figwidth=0.8\textwidth]{Diagram showing decomposition of
        ComponentX}
    \caption{Decomposition of \texttt{ComponentX}}\label{fig:decomp-componentx}
\end{figure}

Describe the decomposition of \texttt{ComponentX} and how this relates to the
requirements.

\section{Deployment view (UML Deployment diagram)}\label{sec:deployment}
Describe the context diagram for the deployment view.
For example, which protocols are used for communication with external systems
and why?

\begin{figure}[!htp]
    \centering
    %\includegraphics[width=\textwidth]{}
    \missingfigure[figwidth=0.8\textwidth]{Context diagram for the deployment
        view.}
    \caption{Context diagram for the deployment view.}\label{fig:depl_context}
\end{figure}

The primary deployment diagram itself and accompanying explanation.
Pay attention to the parts of the deployment diagram which are crucial for
achieving certain non-functional requirements.
Also discuss any alternative deployments that you considered.

\begin{figure}[!htp]
    \centering
    %\includegraphics[width=\textwidth]{}
    \missingfigure[figwidth=0.8\textwidth]{Primary diagram for the deployment
        view.}
    \caption{Primary diagram for the deployment view.}\label{fig:depl_primary}
\end{figure}

\section{Scenarios}\label{sec:scenarios}
Illustrate how your architecture fulfills the most important data flows.
As a rule of thumb, focus on the scenario of the domain description.
Describe the scenario in terms of architectural components using UML Sequence
diagrams and further explain the most important interactions in text.
Illustrating the scenarios serves as a quick validation of the completeness of
your architecture.
If you notice at this point that for some reason, certain functionality or
qualities are not addressed sufficiently in your architecture, it suffices to
document this, together with a rationale of why this is the case according to
you.
You do not have to further refine you architecture at this point.

\subsection{Scenario 1}
Shortly describe the scenario shown in this subsection.
Show the complete scenario using one or more sequence diagrams.

\begin{figure}[!htp]
    \centering
    \includegraphics[width=\textwidth]{figures/UC6 - Deliver document via e-mail.png}
    %\missingfigure[figwidth=0.8\textwidth]{Sequence diagram scenario 1}
    \caption{The system behavior for the first scenario.
        }\label{fig:seq_scenario1}
\end{figure}

\appendix
\section{Element catalog}\label{app:catalog}
List all components and describe their responsibilities and provided
interfaces.
Per interface, list all methods using a Java-like syntax and describe their
effect and exceptions if any.
List all elements and interfaces alphabetically for ease of navigation.

\subsection{Component 1}
\begin{itemize}
    \item \textbf{Description:} Responsibilities of the component.
    \item \textbf{Super-component:} The direct super-component, if any.
    \item \textbf{Sub-components:} the direct sub-components, if any.
\end{itemize}

\subsubsection*{Provided interfaces}
\begin{itemize}
    \item InterfaceA
    \begin{itemize}
        \item \texttt{returntType1 operation1(ParamType param) throws SomeException}
        \begin{itemize}
            \item Effect: Describe the effect of the operation
            \item Exceptions:
            \begin{itemize}
                \item SomeException: Describe when the exception is thrown.
            \end{itemize}

            \item \texttt{void operation2(ParamType2 param)}
            \begin{itemize}
                \item Effect: Describe the effect of the operation
                \item Exceptions: None
            \end{itemize}
        \end{itemize}
    \end{itemize}

    \item InterfaceB
    \begin{itemize}
        \item \texttt{returntType2 operation3()}
        \begin{itemize}
            \item Effect: Describe the effect of the operation
            \item Exceptions: None
        \end{itemize}
    \end{itemize}
\end{itemize}

\section{Defined data types}\label{app:datatypes}
List and describe all data types defined in your interface specifications. List
them alphabetically for ease of navigation.

\begin{itemize}
	\item \texttt{DeliveryMetaData}: Contains the delivery channel, recipient address and other required delivery information (e.g. Customer Organization name, whether receipt tracking applies and whether it is from a recurring batch) for an individual document.
	\item \texttt{BatchDetails}: Encapsulates details about the batch a Customer Organization intends to upload (e.g. type of documents contained in the batch).
	\item \texttt{BatchMetaData}: Contains information such as the CustomerID, the document type of the batch and when it was received.
	\item \texttt{CustomerOrganizationID}: Corresponds to a certain Customer Organization.
	\item \texttt{Document}: Represents a generated document
	\item \texttt{DownloadLink}: Encoded link which decodes to a JobID.
	\item \texttt{GenerationErrorReport}: Contains information about the document for which generation failed (e.g. CustomerOrganizationID) and the reason for the error.
	\item \texttt{Job}: Encapsulates a RawDataEntry, together with its JobID and BatchID.
	\item \texttt{JobID}: Corresponds to a Job and later the Document generated from it.
	\item \texttt{JobStatusEntry}: Encapsulates document state concerning whether it has been generated, delivered, etc. and relevant time stamps.
    \item \texttt{LoginCredentials}: Used when the user wants to log in. Contains the username, password, function the user wants to log in as and general info about the user attempting the login (e.g. IP address).
    \item \texttt{LoginToken}: Depending on the type of user, authorises the user to perform certain actions within the system.
    \item \texttt{PDSLookupLink}: Encoded link which decodes to a JobID and RecipientID.
    \item \texttt{RawDataBatch}: Contains several RawDataEntries and an indication whether it concerns a recurring batch. 
    \item \texttt{RawDataEntry}: Contains all raw data and meta data necessary such that a document can be generated.
    \item \texttt{RawDataMetaData}: Contains information such as the identifier of the raw data unique within the Customer Organization, the selected delivery channel, the name of the addressee and the address of the addressee (the form of which depends on the selected delivery channel).
    \item \texttt{RegistrationDetails}: Contains details that can be used to register a Recipient such as first name, last name, e-mail address and postal address.
    \item \texttt{Template}: Contains fields that can be filled out and an indication which fields are mandatory.
    \item \texttt{UserSession}: Container for login token; keeps state for logged-in user.
\end{itemize}

\end{document}
