\section{Client-server view (UML Component diagram)}\label{sec:client-server}
The context diagram of the client-server view.
Discuss which components communicate with external components and what these
external components represent.

\begin{figure}[!htp]
    \centering
    %\includegraphics[width=\textwidth]{}
    \missingfigure[figwidth=0.8\textwidth]{Context diagram of the client-server view.}
    \caption{Context diagram for the client-server view.
        }\label{fig:cc-context}
\end{figure}

The primary diagram and accompanying explanation.

\begin{figure}[!htp]
    \centering
    %\includegraphics[width=\textwidth]{}
    \missingfigure[figwidth=0.8\textwidth]{Primary diagram of the client-server
        view.}
    \caption{Primary diagram of the client-server view.}\label{fig:cs-primary}
\end{figure}

\subsection{Main architectural decisions}
Discuss your architectural decisions for the most important requirements in
more detail using the components of the client-server view.
Pay attention to the solutions that you employed and the alternatives that you
considered.
The explanation here must be self-contained and complete.
Imagine you had to describe how the architecture supports the core
functionality to someone that is looking at the client-server view only.
Hide unnecessary details (these should be shown in the decomposition view).

\subsubsection{ReqX\@: requirement name}
Describe the design choices related to \emph{ReqX} together with the rationale
of why these choices where made.

\subsubsection*{Alternatives considered}
\paragraph{Alternative(s) for choice 1} Explain what alternative(s) you
considered for this design choice and why they where not selected.