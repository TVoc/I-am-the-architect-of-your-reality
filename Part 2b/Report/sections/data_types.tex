\section{Defined data types}\label{app:datatypes}
List and describe all data types defined in your interface specifications. List
them alphabetically for ease of navigation.

\begin{itemize}
	\item \texttt{AuthToken}: Authorises any automated system of a Customer Organization or Social Secretary to submit batches of raw data/meta-data on the Customer Organization's behalf.
	\item \texttt{DeliveryMetaData}: Contains the delivery channel, recipient address and other required delivery information (e.g. Customer Organization name, whether receipt tracking applies and whether it is from a recurring batch) for an individual document.
	\item \texttt{BatchDetails}: Encapsulates details about the batch a Customer Organization intends to upload. Examples of such details are:
	\begin{itemize}
		\item Type of documents contained in the batch
		\item Number of entries
		\item Whether this is a recurring batch
		\item Which recurring batch it is, if applicable
		\item Whether it contains corrections of entries that the Customer Organization tried to submit earlier as part of a recurring batch (and from which month)
	\end{itemize}
	\item \texttt{BatchID}: Corresponds to a RawDataBatch
	\item \texttt{BatchMetaData}: Contains information such as the CustomerOrganizationID, the document type of the batch, when it was received and which recurring batch it is (if applicable).
	\item \texttt{CustomerOrganizationID}: Corresponds to a certain Customer Organization. Concretely, the Customer Organization's name serves as identifier.
	\item \texttt{Document}: Represents a generated document
	\item \texttt{DocumentMetaData}: Stored with the corresponding Document in the document database. Includes the JobID, the name of the Customer Organization, the time when it was received and the type of the document and details relevant to delivery (e.g. delivery method and Recipient address).
	\item \texttt{DocumentQueryParameter}: Contains the query parameters of a lookup in the Personal Document Store database, such as type of document, (partially specified) name of the Customer Organization that sent it and date range. The RecipientID of the Recipient is always present.
	\item \texttt{DownloadLink}: Encoded link which decodes to a JobID.
	\item \texttt{GenerationErrorReport}: Contains information about the document for which generation failed (e.g. CustomerOrganizationID) and the reason for the error.
	\item \texttt{Job}: Encapsulates a RawDataEntry, together with its JobID and BatchID.
	\item \texttt{JobID}: Corresponds to a Job and later the Document generated from it.
	\item \texttt{JobStatusEntry}: Encapsulates document state concerning whether it has been generated, delivered, etc. and relevant time stamps.
    \item \texttt{LoginCredentials}: Used when the user wants to log in. Contains the username, password, function the user wants to log in as and general info about the user attempting the login (e.g. IP address).
    \item \texttt{LoginToken}: Depending on the type of user, authorises the user to perform certain actions within the system.
    \item \texttt{PDSLookupLink}: Encoded link which decodes to a JobID and RecipientID.
    \item \texttt{RawDataBatch}: Contains several entries of raw data as supplied by the user, an indication whether it concerns a recurring batch and an indication if it contains corrections of entries in an earlier submitted recurring batch (and from which month). 
    \item \texttt{RawDataEntry}: Contains all raw data and meta data necessary such that a document can be generated.
    \item \texttt{RawDataMetaData}: Contains information such as the identifier of the raw data unique within the Customer Organization, the selected delivery channel, the name of the addressee and the address of the addressee (the form of which depends on the selected delivery channel).
    \item \texttt{RecipientID}: Corresponds to a Registered Recipient. Concretely, it matches the Registered Recipient's e-mail address.
    \item \texttt{RegistrationDetails}: Contains details that can be used to register a Recipient such as first name, last name, e-mail address and postal address.
    \item \texttt{Template}: Contains fields that can be filled out and an indication which fields are mandatory.
    \item \texttt{UserSession}: Container for login token; keeps state for logged-in user (such as IP address in order to, for example, directly send the answer to document lookup queries to the user).
\end{itemize}