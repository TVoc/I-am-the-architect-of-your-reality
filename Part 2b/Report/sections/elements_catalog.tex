\section{Element catalog}\label{app:catalog}
In this section, we provide an alphabetically ordered catalog of all components and the interfaces they offer.

\subsection{ConnectionManager}
\begin{itemize}
	\item \textbf{Description:} The \ttt{ConnectionManager} accepts connections from automated Customer Organization or Social Secretary systems so they may submit raw data batches
	\item \textbf{Super-component:} \ttt{SubmissionSubsystem}.
	\item \textbf{Sub-components:} None
\end{itemize}

\subsubsection*{Provided interfaces}
\begin{itemize}
	\item ExternalRawDataSubmission
	\begin{itemize}
		\item \texttt{void openConnection(AuthToken authToken) throws InvalidAuthTokenException}
		\begin{itemize}
			\item Effect: Opens a connection, after which the caller may submit batches of raw data/meta-data.
			\item Exceptions:
			\begin{itemize}
				\item InvalidAuthTokenException: \ttt{authToken} is not registered in the system.
			\end{itemize}
		\end{itemize}
		
		\item \texttt{void submitPlannedBatchDetails(BatchDetails batchDetails) throws NoConnectionException, InvalidBatchException}
		\begin{itemize}
			\item Effect: Verifies whether the Customer Organization (or Social Secretary acting on the Customer Organization's behalf) may submit a batch specified by \ttt{batchDetails}.
			\item Exceptions:
			\begin{itemize}
				\item NoConnectionException: \texttt{openConnection} was not called earlier
				\item InvalidBatchException: One of the following occurs:
				\begin{enumerate}
					\item The Customer Organization on whose behalf the batch would be submitted may not submit the specified type of documents.
					\item If this concerns a recurring batch, the number of entries contained in the batch to be submitted exceeds the number of allowed entries.
					\item If this concerns a recurring batch, the Customer Organization is trying to submit the batch too early.
					\item If it is indicated this recurring batch contains corrections of entries the Customer Organization tried to submit earlier, there is no such earlier submission of a recurring batch.
					\item If it is indicated this recurring batch contains corrections of entries the Customer Organization tried to submit earlier, the total number of entries submitted as part of that recurring batch would exceed the number of allowed entries.
				\end{enumerate}
			\end{itemize}
		\end{itemize}
		
		\item \texttt{void submitRawData(RawDataBatch batch) throws NoConnectionException, NoCorrespondingPlannedBatchException, BatchVerificationException}
		\begin{itemize}
			\item Effect: Accepts a raw data batch, verifies the individual entries and generates jobs from them
			\item Exceptions:
			\begin{itemize}
				\item NoConnectionException: \texttt{openConnection} was not called earlier.
				\item NoCorrespondingPlannedBatchException: there was no call to \texttt{submitPlannedBatchDetails} with a \texttt{BatchDetails} object that corresponds to the given batch
				\item BatchVerficationException: One of the following occurs:
				\begin{enumerate}
					\item The batch was improperly formatted (for example, the batch was a zipfile that could not be opened)
					\item Some individual entries are incorrect; the thrown exception indicates which entries are incorrect and why. Note that all entries that did pass verification were processed correctly.
				\end{enumerate} 
			\end{itemize}
		\end{itemize}
	\end{itemize}
\end{itemize}

\subsection{DocumentDB}\todo{missing methods for filling PDS}
\begin{itemize}
    \item \textbf{Description:} Stores all documents ever generated by the eDocs system.
    \item \textbf{Super-component:} \ttt{DocumentStorageSubsystem}.
    \item \textbf{Sub-components:} \ttt{DocumentDBReplica}, \ttt{DocumentDBReplicationManager}.
\end{itemize}

\subsubsection*{Provided interfaces}
\begin{itemize}
    \item DocumentManagement
	\begin{itemize}
		\item \texttt{void storeDocument(Document doc, DocumentMetaData metaData)}
		\begin{itemize}
			\item Effect: Stores the specified document and its meta-data in the database.
			\item Exceptions: None
		\end{itemize}

		\item \texttt{void storeDocuments(List<Tuple<Document doc, DocumentMetaData metaData>> docs)}
		\begin{itemize}
			\item Effect: Stores all Documents in \ttt{docs} in the database.
			\item Exceptions: None
		\end{itemize}
		
		\item \texttt{Document fetchDocument(JobID docID) throws NoSuchDocumentException}
		\begin{itemize}
			\item Effect: Fetches the Document matching \ttt{docID} and returns it.
			\item Exceptions: 
			\begin{itemize}
				\item NoSuchDocumentException: there is no document matching \ttt{docID}.
			\end{itemize}
		\end{itemize}

		\item \texttt{List<Tuple<Document, DocumentMetaData>> getAllDocumentsFor(DeliveryMethod deliveryMethod, DeliveryAddress deliveryAddress) throws NoSuchRecipientException}
		\begin{itemize}
			\item Effect: Fetches the Documents and their corresponding DocumentMetaData for the Recipient specified by \ttt{deliveryAddress}. What type of address is checked for is determined by \ttt{deliveryMethod}
			\item Exceptions: 
			\begin{itemize}
				\item NoSuchReciopientException: there are no documents for the Recipient specified by \ttt{RecipientAddress}.
			\end{itemize}
		\end{itemize}
	\end{itemize}

    \item DownloadLookup
	\begin{itemize}
		\item \texttt{void fetchDocument(JobID docID, UserSession session) throws NoSuchDocumentException}
		\begin{itemize}
			\item Effect: Fetches the Document matching \ttt{docID} and returns it directly to the user's software via \ttt{session}.
			\item Exceptions:
			\begin{itemize}
				\item NoSuchDocumentException: there is no document matching \ttt{docID}. This exception is propagated directly to the user's software via \ttt{session}.
			\end{itemize}
		\end{itemize}
	\end{itemize}
\end{itemize}

\subsection{DocumentDBReplica}
\begin{itemize}
    \item \textbf{Description:} Stores all documents ever generated by the eDocs system. Must answer to pings lest it be marked as failed.
    \item \textbf{Super-component:} The direct super-component, if any.
    \item \textbf{Sub-components:} the direct sub-components, if any.
\end{itemize}

\subsubsection*{Provided interfaces}
\begin{itemize}
    \item DocumentManagement
	\begin{itemize}
		\item \texttt{void storeDocument(Document doc, DocumentMetaData metaData)}
		\begin{itemize}
			\item Effect: Stores the specified document and its meta-data in the database.
			\item Exceptions: None
		\end{itemize}

		\item \texttt{void storeDocuments(List<Tuple<Document doc, DocumentMetaData metaData>> docs)}
		\begin{itemize}
			\item Effect: Stores all Documents in \ttt{docs} in the database.
			\item Exceptions: None
		\end{itemize}
		
		\item \texttt{Document fetchDocument(JobID docID) throws NoSuchDocumentException}
		\begin{itemize}
			\item Effect: Fetches the Document matching \ttt{docID} and returns it.
			\item Exceptions: 
			\begin{itemize}
				\item NoSuchDocumentException: there is no document matching \ttt{docID}.
			\end{itemize}
		\end{itemize}

		\item \texttt{List<Tuple<Document, DocumentMetaData>> getAllDocumentsFor(DeliveryMethod deliveryMethod, DeliveryAddress deliveryAddress) throws NoSuchRecipientException}
		\begin{itemize}
			\item Effect: Fetches the Documents and their corresponding DocumentMetaData for the Recipient specified by \ttt{deliveryAddress}. What type of address is checked for is determined by \ttt{deliveryMethod}
			\item Exceptions: 
			\begin{itemize}
				\item NoSuchReciopientException: there are no documents for the Recipient specified by \ttt{RecipientAddress}.
			\end{itemize}
		\end{itemize}
	\end{itemize}

    \item DownloadLookup
	\begin{itemize}
		\item \texttt{void fetchDocument(JobID docID, UserSession session) throws NoSuchDocumentException}
		\begin{itemize}
			\item Effect: Fetches the Document matching \ttt{docID} and returns it directly to the user's software via \ttt{session}.
			\item Exceptions:
			\begin{itemize}
				\item NoSuchDocumentException: there is no document matching \ttt{docID}. This exception is propagated directly to the user's software via \ttt{session}.
			\end{itemize}
		\end{itemize}
	\end{itemize}

	\item Ping
	\begin{itemize}
		\item \texttt{Echo ping()}
		\begin{itemize}
			\item Effect: Returns an Echo to reassure the caller that the callee is still available.
			\item Exceptions: None.
		\end{itemize}
	\end{itemize}
\end{itemize}

\subsection{DocumentDBReplicationManager}
\begin{itemize}
    \item \textbf{Description:} Manages its \ttt{DocumentDBReplica}s. Must ping the \ttt{DocumentDBReplica}s upon suspicion of failure and issue an error notification upon timeout of ping.
    \item \textbf{Super-component:} \ttt{DocumentDB}.
    \item \textbf{Sub-components:} None.
\end{itemize}

\subsubsection*{Provided interfaces}
\begin{itemize}
    \item DocumentManagement
	\begin{itemize}
		\item \texttt{void storeDocument(Document doc, DocumentMetaData metaData)}
		\begin{itemize}
			\item Effect: Stores the specified document and its meta-data in the database.
			\item Exceptions: None
		\end{itemize}

		\item \texttt{void storeDocuments(List<Tuple<Document doc, DocumentMetaData metaData>> docs)}
		\begin{itemize}
			\item Effect: Stores all Documents in \ttt{docs} in the database.
			\item Exceptions: None
		\end{itemize}
		
		\item \texttt{Document fetchDocument(JobID docID) throws NoSuchDocumentException}
		\begin{itemize}
			\item Effect: Fetches the Document matching \ttt{docID} and returns it.
			\item Exceptions: 
			\begin{itemize}
				\item NoSuchDocumentException: there is no document matching \ttt{docID}.
			\end{itemize}
		\end{itemize}

		\item \texttt{List<Tuple<Document, DocumentMetaData>> getAllDocumentsFor(DeliveryMethod deliveryMethod, DeliveryAddress deliveryAddress) throws NoSuchRecipientException}
		\begin{itemize}
			\item Effect: Fetches the Documents and their corresponding DocumentMetaData for the Recipient specified by \ttt{deliveryAddress}. What type of address is checked for is determined by \ttt{deliveryMethod}
			\item Exceptions: 
			\begin{itemize}
				\item NoSuchReciopientException: there are no documents for the Recipient specified by \ttt{RecipientAddress}.
			\end{itemize}
		\end{itemize}
	\end{itemize}

    \item DownloadLookup
	\begin{itemize}
		\item \texttt{void fetchDocument(JobID docID, UserSession session) throws NoSuchDocumentException}
		\begin{itemize}
			\item Effect: Fetches the Document matching \ttt{docID} and returns it directly to the user's software via \ttt{session}.
			\item Exceptions:
			\begin{itemize}
				\item NoSuchDocumentException: there is no document matching \ttt{docID}. This exception is propagated directly to the user's software via \ttt{session}.
			\end{itemize}
		\end{itemize}
	\end{itemize}
\end{itemize}

\subsection{DocumentStorageHandler} 
\begin{itemize}
    \item \textbf{Description:} Receives documents to be stored and stores them in \ttt{DocumentDB} and also in \ttt{PDSDB}, if directed to.
    \item \textbf{Super-component:} \ttt{DocumentStorageSubsystem}.
    \item \textbf{Sub-components:} \ttt{GeneratedDocumentManager}, \ttt{PDSCache}.
\end{itemize}

\subsubsection*{Provided interfaces}
\begin{itemize}
    \item DocumentStorage
	\begin{itemize}
		\item \texttt{void storeDocument(Document doc, BatchMetaData batchData, RawDataMetaData rawData, boolean toPDS)}
		\begin{itemize}
			\item Effect: Constructs a \ttt{DocumentMetaData} object from \ttt{batchData} and \ttt{rawData} and stores the given document in the \ttt{DocumentDB} (and also the \ttt{PDSDB} if \ttt{toPDS} is true).
			\item Exceptions: None
		\end{itemize}
		
		\item \texttt{Document fetchDocument(JobID docID) throws NoSuchDocumentException}
		\begin{itemize}
			\item Effect: Fetches the Document matching \ttt{docID} from the \ttt{DocumentDB} and returns it together with its meta-data.
			\item Exceptions: 
			\begin{itemize}
				\item NoSuchDocumentException: there is no document matching \ttt{docID}.
			\end{itemize}
		\end{itemize}
	\end{itemize}

	\item PDSContentManagement
	\begin{itemize}
		\item \texttt{void fillPDS(RecipientID recipient) throws NoSuchRecipientException}
		\begin{itemize}
			\item Effect: Fetches all documents addressed to the specified Recipient and stores them in the \ttt{PDSDB}.
			\item Exceptions:
			\begin{itemize}
				\item NoSuchRecipientException: There are no documents stored in the document storage that are addressed to the specified Recipient.
			\end{itemize}
		\end{itemize}
				
		\item \texttt{void purgePDSOfDocumentsOf(RecipientID recipient) throws NoSuchRecipientException}
		\begin{itemize}
			\item Effect: Removes all documents addressed to the specified Recipient from the \ttt{PDSDB}.
			\item Exceptions:
			\begin{itemize}
				\item NoSuchRecipientException: There are no documents stored in the Personal Document Store database that are addressed to the specified Recipient.
			\end{itemize}
		\end{itemize}
	\end{itemize}
\end{itemize}

\subsection{DocumentStorageSubsystem}
\begin{itemize}
	\item \textbf{Description:} Responsible for storing documents and answering lookup queries.
	\item \textbf{Super-component:} None.
	\item \textbf{Sub-components:} \ttt{DocumentDB}, \ttt{DocumentStorageHandler}, \ttt{PDSDB}.
\end{itemize}

\subsubsection*{Provided interfaces}
\begin{itemize}
	\item DocumentStorage
	\begin{itemize}
		\item \texttt{void storeDocument(Document doc, BatchMetaData batchData, RawDataMetaData rawData, boolean toPDS)}
		\begin{itemize}
			\item Effect: Constructs a \ttt{DocumentMetaData} object from \ttt{batchData} and \ttt{rawData} and stores the given document in the document database (and also the Personal Document Store database if \ttt{toPDS} is true).
			\item Exceptions: None
		\end{itemize}

		\item \texttt{void storeDocuments(List<Tuple<Document doc, DocumentMetaData metaData>> docs)}
		\begin{itemize}
			\item Effect: Stores all Documents in \ttt{docs} in the database.
			\item Exceptions: None
		\end{itemize}
		
		\item \texttt{Document fetchDocument(JobID docID) throws NoSuchDocumentException}
		\begin{itemize}
			\item Effect: Fetches the Document matching \ttt{docID} from the database and returns it.
			\item Exceptions: 
			\begin{itemize}
				\item NoSuchDocumentException: there is no document matching \ttt{docID}.
			\end{itemize}
		\end{itemize}
	\end{itemize}
	
	\item DownloadLookup
	\begin{itemize}
		\item \texttt{void fetchDocument(JobID docID, UserSession session) throws NoSuchDocumentException}
		\begin{itemize}
			\item Effect: Fetches the Document matching \ttt{docID} from the database and returns it directly to the user's software via \ttt{session}.
			\item Exceptions:
			\begin{itemize}
				\item NoSuchDocumentException: there is no document matching \ttt{docID}. This exception is propagated directly to the user's software via \ttt{session}.
			\end{itemize}
		\end{itemize}
	\end{itemize}
	
	\item PDSContentManagement
	\begin{itemize}
		\item \texttt{void fillPDSWithDocumentsOf(RecipientID recipient) throws NoSuchRecipientException}
		\begin{itemize}
			\item Effect: Fetches all documents addressed to the specified Recipient and stores them in the Personal Document Store database
			\item Exceptions:
			\begin{itemize}
				\item NoSuchRecipientException: There are no documents stored in the document storage that are addressed to the specified Recipient.
			\end{itemize}
		\end{itemize}
				
		\item \texttt{void purgePDSOfDocumentsOf(RecipientID recipient) throws NoSuchRecipientException}
		\begin{itemize}
			\item Effect: Removes all documents addressed to the specified Recipient from the Personal Document Store database.
			\item Exceptions:
			\begin{itemize}
				\item NoSuchRecipientException: There are no documents stored in the Personal Document Store database that are addressed to the specified Recipient.
			\end{itemize}
		\end{itemize}
	\end{itemize}
	
	\item PDSLookup
	\begin{itemize}
		\item \texttt{void fetchDocument(JobID docID, UserSession session) throws NoSuchDocumentException}
		\begin{itemize}
			\item Effect: Fetches the document specified by \ttt{docID} and returns it directly to the user's software via \ttt{session}.
			\item Exceptions:
			\begin{itemize}
				\item NoSuchDocumentException: The document specified by \ttt{docID} does not exist. This exception is propagated directly to the user's software via \ttt{session}.
			\end{itemize}
		\end{itemize}
				
		\item \texttt{void performLookupQuery(DocumentQueryParameter query, UserSession session)}
		\begin{itemize}
			\item Effect: Computes which of the Recipient's documents matches \ttt{query} and returns the resulting list of \ttt{DocumentMetaData} objects directly to the user's software via \ttt{session}. Note that this list may be empty if none of the Recipient's documents match.
			\item Exceptions: None.
		\end{itemize}
	\end{itemize}

	\item Ping
	\begin{itemize}
		\item \texttt{Echo ping()}
		\begin{itemize}
			\item Effect: Returns an Echo to reassure the caller that the callee is still available.
			\item Exceptions: None.
		\end{itemize}
	\end{itemize}
\end{itemize}

\subsection{DownloadLookupModule}
\begin{itemize}
    \item \textbf{Description:} Responsible for forwarding requests for lookups via unique download link.
    \item \textbf{Super-component:} \ttt{LookupSubsystem}.
    \item \textbf{Sub-components:} None.
\end{itemize}

\subsubsection*{Provided interfaces}
\begin{itemize}
    \item LookupManagement
    \begin{itemize}
        \item \texttt{void lookupDownloadLink(DownloadLink docLink, UserSession session) throws InvalidOrExpiredDownloadLinkException}
        \begin{itemize}
            \item Effect: Looks up the document corresponding to \ttt{docLink} and returns it directly to the user's software via \ttt{session}.
            \item Exceptions:
            \begin{itemize}
                \item InvalidOrExpiredDownloadLinkException: Either the link is invalid or the link has expired.
            \end{itemize}
        \end{itemize}
    \end{itemize}
\end{itemize}

\subsection{DownloadLinkCatalogue}
\begin{itemize}
    \item \textbf{Description:} Responsible for storing unique download links for 30 days and notifying the Customer Organization whenever a link expires without the Recipient ever having viewed the document.
    \item \textbf{Super-component:} \ttt{LookupSubsystem}.
    \item \textbf{Sub-components:} None.
\end{itemize}

\subsubsection*{Provided interfaces}
\begin{itemize}
    \item DownloadLinkManagement
    \begin{itemize}
        \item \texttt{boolean linkIsValid(DownloadLink link)}
        \begin{itemize}
            \item Effect: Indicates whether the specified link is present in the \ttt{DownloadLinkCatalogue}.
            \item Exceptions: None
        \end{itemize}

        \item \texttt{void storeLink(DownloadLink link)}
		    \begin{itemize}
                \item Effect: Stores the specified \ttt{DownloadLink}.
                \item Exceptions: None
            \end{itemize}
    \end{itemize}
\end{itemize}

\subsection{GeneratedDocumentManager}
\begin{itemize}
    \item \textbf{Description:} Responsible for storing documents in the \ttt{DocumentDB} and also the \ttt{PDSDB} (whenever appropriate). If the \ttt{PDSDB} fails, it must store documents in the \ttt{PDSCache} and retrieve them when the \ttt{PDSDB} comes back online.
    \item \textbf{Super-component:} \ttt{DocumentStorageHandler}.
    \item \textbf{Sub-components:} None.
\end{itemize}

\subsubsection*{Provided interfaces}
\begin{itemize}
    \item DocumentStorage
	\begin{itemize}
		\item \texttt{void storeDocument(Document doc, BatchMetaData batchData, RawDataMetaData rawData, boolean toPDS)}
		\begin{itemize}
			\item Effect: Constructs a \ttt{DocumentMetaData} object from \ttt{batchData} and \ttt{rawData} and stores the given document in the \ttt{DocumentDB} (and also the \ttt{PDSDB} if \ttt{toPDS} is true).
			\item Exceptions: None
		\end{itemize}
		
		\item \texttt{Document fetchDocument(JobID docID) throws NoSuchDocumentException}
		\begin{itemize}
			\item Effect: Fetches the Document matching \ttt{docID} from the \ttt{DocumentDB} and returns it together with its meta-data.
			\item Exceptions: 
			\begin{itemize}
				\item NoSuchDocumentException: there is no document matching \ttt{docID}.
			\end{itemize}
		\end{itemize}
	\end{itemize}

	\item PDSContentManagement
	\begin{itemize}
		\item \texttt{void fillPDS(RecipientID recipient) throws NoSuchRecipientException}
		\begin{itemize}
			\item Effect: Fetches all documents addressed to the specified Recipient and stores them in the \ttt{PDSDB}.
			\item Exceptions:
			\begin{itemize}
				\item NoSuchRecipientException: There are no documents stored in the document storage that are addressed to the specified Recipient.
			\end{itemize}
		\end{itemize}
				
		\item \texttt{void purgePDSOfDocumentsOf(RecipientID recipient) throws NoSuchRecipientException}
		\begin{itemize}
			\item Effect: Removes all documents addressed to the specified Recipient from the \ttt{PDSDB}.
			\item Exceptions:
			\begin{itemize}
				\item NoSuchRecipientException: There are no documents stored in the Personal Document Store database that are addressed to the specified Recipient.
			\end{itemize}
		\end{itemize}
	\end{itemize}
\end{itemize}

\subsection{JobGenerator}
\begin{itemize}
	\item \textbf{Description:} Accepts verified batches of raw data and generates jobs from them
	\item \textbf{Super-component:} \ttt{SubmissionSubsystem}.
	\item \textbf{Sub-components:} None.
\end{itemize}

\subsubsection*{Provided interfaces}
\begin{itemize}
	\item JobDataSubmission
	\begin{itemize}
		\item \texttt{void generateJobs(RawDataBatch dataBatch)}
		\begin{itemize}
			\item Effect: Generates \ttt{RawDataEntry} and \ttt{JobID} objects from each individual entry in \ttt{dataBatch}. Additionally, generates \ttt{BatchMetaData} and \ttt{BatchID} objects and associates each \ttt{RawDataEntry} with that \ttt{BatchID} by generating a \ttt{Job} object from the \ttt{JobID}, \ttt{RawDataEntry} and \ttt{BatchID}. After this is completed, submits all generated \ttt{Job} objects and the \ttt{BatchMetaData} to the \ttt{JobStorageSubsystem}
			\item Exceptions: None
		\end{itemize}
	\end{itemize}
\end{itemize}

\subsection{LoadBalancer}
\begin{itemize}
    \item \textbf{Description:} Responsible for load balancing of lookup requests. Receives information about document database processing capabilities in order to achieve this.
    \item \textbf{Super-component:} The direct super-component, if any.
    \item \textbf{Sub-components:} the direct sub-components, if any.
\end{itemize}

\subsubsection*{Provided interfaces}
\begin{itemize}
    \item LoadInfoManagement
    \begin{itemize}
        \item \texttt{void notifyOfSystemCapabilities(CapabilityReport report)}
        \begin{itemize}
            \item Effect: Notifies of the current processing power available to the document databases upon which decisions about load balancing may be made.
            \item Exceptions: None.
        \end{itemize}
    \end{itemize}

    \item InterfaceB
    \begin{itemize}
        \item \texttt{returntType2 operation3()}
        \begin{itemize}
            \item Effect: Describe the effect of the operation
            \item Exceptions: None
        \end{itemize}
    \end{itemize}
\end{itemize}

\subsection{LookupLinkCodec}
\begin{itemize}
    \item \textbf{Description:} Responsible for encoding links (and passing them to the \ttt{DownloadLinkCatalogue}, if it is a unique download link) and decoding links.
    \item \textbf{Super-component:} \ttt{LookupSubsystem}.
    \item \textbf{Sub-components:} None.
\end{itemize}

\subsubsection*{Provided interfaces}
\begin{itemize}
    \item LookupLinkManagement
    \begin{itemize}
        \item \texttt{DownloadLink generateDownloadLink(JobID docID)}
        \begin{itemize}
            \item Effect: Generates a download link that encodes \ttt{docID} and returns it, also storing it for 30 days.
            \item Exceptions: None
        \end{itemize}

        \item \texttt{JobID decodeDownloadLink(DownloadLink link)}
		    \begin{itemize}
                \item Effect: Decodes the given link and returns the decoded \ttt{JobID}.
                \item Exceptions: None
            \end{itemize}

		\item \texttt{Tuple<RecipientID, JobID> decodeLink(PDSLookupLink link)}
		    \begin{itemize}
                \item Effect: Decodes the given link and returns the decoded \ttt{JobID} and \ttt{RecipientID}.
                \item Exceptions: None
            \end{itemize}
    \end{itemize}
\end{itemize}

\subsection{LookupSubsystem}
\begin{itemize}
    \item \textbf{Description:} Responsible for forwarding requests for documents to the \ttt{DocumentStorageSubsystem}. Must handle load balancing of lookups and management of download links.
    \item \textbf{Super-component:} None.
    \item \textbf{Sub-components:} \ttt{DownloadLookupModule} \ttt{DownloadLinkCatalogue}, \ttt{LoadBalancer}, \ttt{LookupLinkCodec}, \ttt{PDSLookupModule}.
\end{itemize}

\subsubsection*{Provided interfaces}
\begin{itemize}
    \item LoadInfoManagement
    \begin{itemize}
        \item \texttt{void notifyOfSystemCapabilities(CapabilityReport report)}
        \begin{itemize}
            \item Effect: Notifies of the current processing power available to the document databases upon which decisions about load balancing may be made.
            \item Exceptions: None.
        \end{itemize}
    \end{itemize}

    \item LookupManagement
    \begin{itemize}
        \item \texttt{void lookupDownloadLink(DownloadLink docLink, UserSession session) throws InvalidOrExpiredDownloadLinkException}
        \begin{itemize}
            \item Effect: Looks up the document corresponding to \ttt{docLink} and returns it directly to the user's software via \ttt{session}.
            \item Exceptions:
            \begin{itemize}
                \item InvalidOrExpiredDownloadLinkException: Either the link is invalid or the link has expired.
            \end{itemize}
        \end{itemize}
    \end{itemize}

	\item LookupLinkManagement
    \begin{itemize}
        \item \texttt{DownloadLink generateDownloadLink(JobID docID)}
        \begin{itemize}
            \item Effect: Generates a download link that encodes \ttt{docID} and returns it, also storing it for 30 days.
            \item Exceptions: None
        \end{itemize}

        \item \texttt{JobID decodeDownloadLink(DownloadLink link)}
		    \begin{itemize}
                \item Effect: Decodes the given link and returns the decoded \ttt{JobID}.
                \item Exceptions: None
            \end{itemize}

		\item \texttt{Tuple<RecipientID, JobID> decodeLink(PDSLookupLink link)}
		    \begin{itemize}
                \item Effect: Decodes the given link and returns the decoded \ttt{JobID} and \ttt{RecipientID}.
                \item Exceptions: None
            \end{itemize}
    \end{itemize}

	\item PDSLookupManagement
    \begin{itemize}
        \item \texttt{void lookupDoc(JobID docID, UserSession session) throws NoSuchDocumentException}
        \begin{itemize}
            \item Effect: Looks up the document corresponding to \ttt{docID} in the \ttt{PDSDB} and returns it directly to the user's software via \ttt{session}.
            \item Exceptions: 
			\begin{itemize}
				\item NoSuchDocumentException: There is no document corresponding to \ttt{docID} in the \ttt{PDSDB}. This exception is propagated directly to the user via \ttt{session}.
			\end{itemize}
        \end{itemize}

        \item \texttt{void lookupDocWithLink(PDSLookupLink link, UserSession session) throws UnauthorizedLookupException, NoSuchDocumentException}
		    \begin{itemize}
                \item Effect: Decodes the given link to a \ttt{JobID} and \ttt{RecipientID} and returns the corresponding \ttt{Document} directly to the user's software via \ttt{session}.
                \item Exceptions: 
				\begin{itemize}
					\item NoSuchDocumentException: There is no document corresponding to the \ttt{JobID} encoded in \ttt{link} in the \ttt{PDSDB}. This exception is propagated directly to the user via \ttt{session}.
					\item UnauthorizedLookupException: The \ttt{RecipientID} encoded in \ttt{link} does not correspond with the \ttt{RecipientID} contained the \ttt{session}.
				\end{itemize}
            \end{itemize}
    \end{itemize}
\end{itemize}

\subsection{PDSCache}
\begin{itemize}
    \item \textbf{Description:} Responsible for caching documents meant to be stored in the \ttt{PDSDB} whenever the latter fails. Must be able to store at least three hours' worth of documents.
    \item \textbf{Super-component:} \ttt{DocumentStorageHandler}.
    \item \textbf{Sub-components:} None.
\end{itemize}

\subsubsection*{Provided interfaces}
\begin{itemize}
    \item PDSCacheManagement
    \begin{itemize}
        \item \texttt{void cacheDocument(Document document, DocumentMetaData data) throws SomeException}
        \begin{itemize}
            \item Effect: Stores the specified \ttt{Document} together with its \ttt{DocumentMetaData} in the cache.
            \item Exceptions: None
        \end{itemize}

        \item \texttt{List<Tuple<Document, DocumentMetaData>> flushCache()}
		    \begin{itemize}
                \item Effect: Retrieves all documents stored in the cache and removes them from the cache.
                \item Exceptions: None
            \end{itemize}
    \end{itemize}
\end{itemize}

\subsection{PDSDB}
\begin{itemize}
    \item \textbf{Description:} Responsible for storing all Personal Document Store documents.
    \item \textbf{Super-component:} \ttt{DocumentStorageSubsystem}.
    \item \textbf{Sub-components:} \ttt{PDSLongTermDocumentManager}, \ttt{PDSReplicationManager}, \ttt{PDSDBReplica}.
\end{itemize}

\subsubsection*{Provided interfaces}
\begin{itemize}
	\item PDSDocumentManagement
	\begin{itemize}
		\item \texttt{void storeDocument(Document doc, DocumentMetaData metaData)}
		\begin{itemize}
			\item Effect: Stores the specified document and its meta-data in the database.
			\item Exceptions: None
		\end{itemize}

		\item \texttt{void storeDocuments(List<Tuple<Document doc, DocumentMetaData metaData>> docs)}
		\begin{itemize}
			\item Effect: Stores all Documents in \ttt{docs} in the database.
			\item Exceptions: None
		\end{itemize}

		\item \texttt{void purgePDSOfDocumentsOf(RecipientID recipient) throws NoSuchRecipientException}
		\begin{itemize}
			\item Effect: Removes all documents addressed to the specified Recipient from the Personal Document Store database.
			\item Exceptions:
			\begin{itemize}
				\item NoSuchRecipientException: There are no documents stored in the Personal Document Store database that are addressed to the specified Recipient.
			\end{itemize}
		\end{itemize}
	\end{itemize}

    \item PDSLookup
	\begin{itemize}
		\item \texttt{void fetchDocument(JobID docID, UserSession session) throws NoSuchDocumentException}
		\begin{itemize}
			\item Effect: Fetches the document specified by \ttt{docID} and returns it directly to the user's software via \ttt{session}.
			\item Exceptions:
			\begin{itemize}
				\item NoSuchDocumentException: The document specified by \ttt{docID} does not exist. This exception is propagated directly to the user's software via \ttt{session}.
			\end{itemize}
		\end{itemize}
				
		\item \texttt{void performLookupQuery(DocumentQueryParameter query, UserSession session)}
		\begin{itemize}
			\item Effect: Computes which of the Recipient's documents matches \ttt{query} and returns the resulting list of \ttt{DocumentMetaData} objects directly to the user's software via \ttt{session}. Note that this list may be empty if none of the Recipient's documents match.
			\item Exceptions: None.
		\end{itemize}
	\end{itemize}

	\item Ping
	\begin{itemize}
		\item \texttt{Echo ping()}
		\begin{itemize}
			\item Effect: Returns an Echo to reassure the caller that the callee is still available.
			\item Exceptions: None.
		\end{itemize}
	\end{itemize}
\end{itemize}

\subsection{PDSDBReplica} \todo{write getDocumentsFailedSince}
\begin{itemize}
    \item \textbf{Description:} Stores all documents that belong in the Personal Document Store. Also must respond to pings when directed to by the \ttt{PDSReplicaManager}
    \item \textbf{Super-component:} \ttt{PDSDB}.
    \item \textbf{Sub-components:} None.
\end{itemize}

\subsubsection*{Provided interfaces}
\begin{itemize}
    \item ExtendedDocumentReplicaMgmt
	\begin{itemize}
		\item \texttt{void storeDocument(Document doc, DocumentMetaData metaData)}
		\begin{itemize}
			\item Effect: Stores the specified document and its meta-data in the database.
			\item Exceptions: None
		\end{itemize}

		\item \texttt{void storeDocuments(List<Tuple<Document doc, DocumentMetaData metaData>> docs)}
		\begin{itemize}
			\item Effect: Stores all Documents in \ttt{docs} in the database.
			\item Exceptions: None
		\end{itemize}

		\item \texttt{Tuple<Document, DocumentMetaData> fetchAndRemoveOldDocuments()}
		\begin{itemize}
			\item Effect: Fetches all documents that both have been stored for at least 30 days and have been received and removes them.
			\item Exceptions: None
		\end{itemize}

		\item \texttt{Document fetchDocument(JobID docID) throws NoSuchDocumentException}
		\begin{itemize}
			\item Effect: Fetches the document specified by \ttt{docID} and returns it.
			\item Exceptions:
			\begin{itemize}
				\item NoSuchDocumentException: The document specified by \ttt{docID} does not exist.
			\end{itemize}
		\end{itemize}
				
		\item \texttt{List<DocumentMetaData> performLookupQuery(DocumentQueryParameter query, UserSession session)}
		\begin{itemize}
			\item Effect: Computes which of the Recipient's documents matches \ttt{query} and returns the resulting list of \ttt{DocumentMetaData} objects. Note that this list may be empty if none of the Recipient's documents match.
			\item Exceptions: None.
		\end{itemize}

		\item \texttt{void purgePDSOfDocumentsOf(RecipientID recipient) throws NoSuchRecipientException}
		\begin{itemize}
			\item Effect: Removes all documents addressed to the specified Recipient from the Personal Document Store database.
			\item Exceptions:
			\begin{itemize}
				\item NoSuchRecipientException: There are no documents stored in the Personal Document Store database that are addressed to the specified Recipient.
			\end{itemize}
		\end{itemize}

		\item \texttt{List<Document, DocumentMetaData> getDocumentsSince(TimeStamp time)}
		\begin{itemize}
			\item Effect: Retrieves all documents that have been stored since the given \ttt{TimeStamp}.
			\item Exceptions: None.
		\end{itemize}
	\end{itemize}

	\item Ping
	\begin{itemize}
		\item \texttt{Echo ping()}
		\begin{itemize}
			\item Effect: Returns an Echo to reassure the caller that the callee is still available.
			\item Exceptions: None.
		\end{itemize}
	\end{itemize}
\end{itemize}

\subsection{PDSLongTermDocumentHandler}
\begin{itemize}
    \item \textbf{Description:} The first component to receive requests to store and look up documents in the Personal Document Store. Transfers documents that have been both read and are older than 30 days from the new document storage to the old document storage once a day.
    \item \textbf{Super-component:} \ttt{PDSDB}.
    \item \textbf{Sub-components:} None.
\end{itemize}

\subsubsection*{Provided interfaces}
\begin{itemize}
    \item PDSLookup
	\begin{itemize}
		\item \texttt{void fetchDocument(JobID docID, UserSession session) throws NoSuchDocumentException}
		\begin{itemize}
			\item Effect: Fetches the document specified by \ttt{docID} and returns it directly to the user's software via \ttt{session}.
			\item Exceptions:
			\begin{itemize}
				\item NoSuchDocumentException: The document specified by \ttt{docID} does not exist. This exception is propagated directly to the user's software via \ttt{session}.
			\end{itemize}
		\end{itemize}
				
		\item \texttt{void performLookupQuery(DocumentQueryParameter query, UserSession session)}
		\begin{itemize}
			\item Effect: Computes which of the Recipient's documents matches \ttt{query} and returns the resulting list of \ttt{DocumentMetaData} objects directly to the user's software via \ttt{session}. Note that this list may be empty if none of the Recipient's documents match.
			\item Exceptions: None.
		\end{itemize}
	\end{itemize}

    \item PDSDocumentManagement
	\begin{itemize}
		\item \texttt{void storeDocument(Document doc, DocumentMetaData metaData)}
		\begin{itemize}
			\item Effect: Stores the specified document and its meta-data in the cluster for new documents.
			\item Exceptions: None
		\end{itemize}

		\item \texttt{void storeDocuments(List<Tuple<Document doc, DocumentMetaData metaData>> docs)}
		\begin{itemize}
			\item Effect: Stores all Documents in \ttt{docs} in the database.
			\item Exceptions: None
		\end{itemize}

		\item \texttt{void purgePDSOfDocumentsOf(RecipientID recipient) throws NoSuchRecipientException}
		\begin{itemize}
			\item Effect: Removes all documents addressed to the specified Recipient from the Personal Document Store database.
			\item Exceptions:
			\begin{itemize}
				\item NoSuchRecipientException: There are no documents stored in the Personal Document Store database that are addressed to the specified Recipient.
			\end{itemize}
		\end{itemize}
	\end{itemize}

	\item Ping
	\begin{itemize}
		\item \texttt{Echo ping()}
		\begin{itemize}
			\item Effect: Returns an Echo to reassure the caller that the callee is still available.
			\item Exceptions: None.
		\end{itemize}
	\end{itemize}
\end{itemize}

\subsection{PDSReplicationManager}
\begin{itemize}
    \item \textbf{Description:} Manages its \ttt{PDSDBReplica}s, sending an error notification in case any of them fails.
    \item \textbf{Super-component:} \ttt{PDSDB}.
    \item \textbf{Sub-components:} None.
\end{itemize}

\subsubsection*{Provided interfaces}
\begin{itemize}
    \item ExtendedDocumentMgmt
	\begin{itemize}
		\item \texttt{void storeDocument(Document doc, DocumentMetaData metaData)}
		\begin{itemize}
			\item Effect: Stores the specified document and its meta-data in the database.
			\item Exceptions: None
		\end{itemize}

		\item \texttt{void storeDocuments(List<Tuple<Document doc, DocumentMetaData metaData>> docs)}
		\begin{itemize}
			\item Effect: Stores all Documents in \ttt{docs} in the database.
			\item Exceptions: None
		\end{itemize}

		\item \texttt{Tuple<Document, DocumentMetaData> fetchAndRemoveOldDocuments()}
		\begin{itemize}
			\item Effect: Fetches all documents that both have been stored for at least 30 days and have been received and removes them.
			\item Exceptions: None
		\end{itemize}

		\item \texttt{Document fetchDocument(JobID docID) throws NoSuchDocumentException}
		\begin{itemize}
			\item Effect: Fetches the document specified by \ttt{docID} and returns it. Also flags the concerned Document as having been read so that it can be transferred to the storage for old documents once the Document is at least 30 days old.
			\item Exceptions:
			\begin{itemize}
				\item NoSuchDocumentException: The document specified by \ttt{docID} does not exist.
			\end{itemize}
		\end{itemize}
				
		\item \texttt{List<DocumentMetaData> performLookupQuery(DocumentQueryParameter query)}
		\begin{itemize}
			\item Effect: Computes which of the Recipient's documents matches \ttt{query} and returns the resulting list of \ttt{DocumentMetaData} objects. Note that this list may be empty if none of the Recipient's documents match.
			\item Exceptions: None.
		\end{itemize}

		\item \texttt{void purgePDSOfDocumentsOf(RecipientID recipient) throws NoSuchRecipientException}
		\begin{itemize}
			\item Effect: Removes all documents addressed to the specified Recipient from the Personal Document Store database.
			\item Exceptions:
			\begin{itemize}
				\item NoSuchRecipientException: There are no documents stored in the Personal Document Store database that are addressed to the specified Recipient.
			\end{itemize}
		\end{itemize}
	\end{itemize}

	\item Ping
	\begin{itemize}
		\item \texttt{Echo ping()}
		\begin{itemize}
			\item Effect: Returns an Echo to reassure the caller that the callee is still available.
			\item Exceptions: None.
		\end{itemize}
	\end{itemize}
\end{itemize}

\subsection{RawDataVerifier}\label{sec:rawdataverifier}
\begin{itemize}
	\item \textbf{Description:} Verifies raw data batches and their individual entries.
	\item \textbf{Super-component:} \ttt{SubmissionSubsystem}.
	\item \textbf{Sub-components:} None.
\end{itemize}

\subsubsection*{Provided interfaces}
\begin{itemize}
	\item InternalRawDataSubmission
	\begin{itemize}
		\item \texttt{void submitPlannedBatchDetails(BatchDetails batchDetails) throws InvalidBatchException}
		\begin{itemize}
			\item Effect: Verifies whether the Customer Organization (or Social Secretary acting on the Customer Organization's behalf) may submit a batch specified by \ttt{batchDetails}.
			\item Exceptions:
			\begin{itemize}
				\item InvalidBatchException: One of the following occurs:
				\begin{enumerate}
					\item The Customer Organization on whose behalf the batch would be submitted may not submit the specified type of documents.
					\item If this concerns a recurring batch, the number of entries contained in the batch to be submitted exceeds the number of allowed entries.
					\item If this concerns a recurring batch, the Customer Organization is trying to submit the batch too early.
					\item If it is indicated this recurring batch contains corrections of entries the Customer Organization tried to submit earlier, there is no such earlier submission of a recurring batch.
					\item If it is indicated this recurring batch contains corrections of entries the Customer Organization tried to submit earlier, the total number of entries submitted as part of that recurring batch would exceed the number of allowed entries.
				\end{enumerate}
			\end{itemize}
		\end{itemize}
		
		\item \texttt{void submitRawData(RawDataBatch batch) throws NoCorrespondingPlannedBatchException, BatchVerificationException}
		\begin{itemize}
			\item Effect: Verifies the internal consistency of each individual entry and passes the batch to the \ttt{JobGenerator} (minus the entries that did not pass verification). External databases are consulted in order to achieve this (e.g. database of cities and their postal codes). The checks on internal consistency include the following:
			\begin{itemize}
				\item The address provided is consistent with the selected delivery method (e.g. e-mail address with e-mail delivery)
				\item If postal delivery was selected, the city and postal code must be consistent
			\end{itemize}
			\item Exceptions:
			\begin{itemize}
				\item NoCorrespondingPlannedBatchException: there was no call to \texttt{submitPlannedBatchDetails} with a \texttt{BatchDetails} object that corresponds to the given batch
				\item BatchVerficationException: One of the following occurs:
				\begin{enumerate}
					\item The batch was improperly formatted (for example, the batch was a zipfile that could not be opened)
					\item Some individual entries are incorrect; the thrown exception indicates which entries are incorrect and why. Note that all entries that did pass verification were processed correctly.
				\end{enumerate} 
			\end{itemize}
		\end{itemize}
	\end{itemize}
\end{itemize}

\subsection{SubmissionSubsystem}
\begin{itemize}
    \item \textbf{Description:} The \ttt{SubmissionSubsystem} is responsible for receiving batches of raw data/meta-data and generating jobs from them. These batches may be both recurring and non-recurring. 
    \item \textbf{Super-component:} None.
    \item \textbf{Sub-components:} \ttt{ConnectionManager}, \ttt{JobGenerator}, \ttt{RawDataVerifier}.
\end{itemize}

\subsubsection*{Provided interfaces}
\begin{itemize}
    \item ExternalRawDataSubmission
	\begin{itemize}
		\item \texttt{void openConnection(AuthToken authToken) throws InvalidAuthTokenException}
			 \begin{itemize}
				 \item Effect: Opens a connection, after which the caller may submit batches of raw data/meta-data.
		         \item Exceptions:
		            \begin{itemize}
		                \item InvalidAuthTokenException: \ttt{authToken} is not registered in the system.
		            \end{itemize}
		     \end{itemize}

	    \item \texttt{void submitPlannedBatchDetails(BatchDetails batchDetails) throws NoConnectionException, InvalidBatchException}
	         \begin{itemize}
	             \item Effect: Verifies whether the Customer Organization (or Social Secretary acting on the Customer Organization's behalf) may submit a batch specified by \ttt{batchDetails}.
	             \item Exceptions:
	                \begin{itemize}
	                	\item NoConnectionException: \texttt{openConnection} was not called earlier
	                	\item InvalidBatchException: One of the following occurs:
	                	\begin{enumerate}
	                		\item The Customer Organization on whose behalf the batch would be submitted may not submit the specified type of documents.
		                	\item If this concerns a recurring batch, the number of entries contained in the batch to be submitted exceeds the number of allowed entries.
		                	\item If this concerns a recurring batch, the Customer Organization is trying to submit the batch too early.
		                	\item If it is indicated this recurring batch contains corrections of entries the Customer Organization tried to submit earlier, there is no such earlier submission of a recurring batch.
		                	\item If it is indicated this recurring batch contains corrections of entries the Customer Organization tried to submit earlier, the total number of entries submitted as part of that recurring batch would exceed the number of allowed entries.
	                	\end{enumerate}
	                \end{itemize}
	            \end{itemize}
	            
	    \item \texttt{void submitRawData(RawDataBatch batch) throws NoConnectionException, NoCorrespondingPlannedBatchException, BatchVerificationException}
	     \begin{itemize}
	     	\item Effect: Accepts a raw data batch, verifies the individual entries and generates jobs from them
	     	\item Exceptions:
	     	\begin{itemize}
	     		\item NoConnectionException: \texttt{openConnection} was not called earlier.
	     		\item NoCorrespondingPlannedBatchException: there was no call to \texttt{submitPlannedBatchDetails} with a \texttt{BatchDetails} object that corresponds to the given batch
	     		\item BatchVerficationException: One of the following occurs:
	     		\begin{enumerate}
	     			\item The batch was improperly formatted (for example, the batch was a zipfile that could not be opened)
	     			\item Some individual entries are incorrect; the thrown exception indicates which entries are incorrect and why. Note that all entries that did pass verification were processed correctly.
	     		\end{enumerate} 
	     	\end{itemize}
	     \end{itemize}
    \end{itemize}
    
    \item InternalRawDataSubmission
    \begin{itemize}
    	\item \texttt{void submitPlannedBatchDetails(BatchDetails batchDetails) throws InvalidBatchException}
    	\begin{itemize}
    		\item Effect: Verifies whether the Customer Organization (or Social Secretary acting on the Customer Organization's behalf) may submit a batch specified by \ttt{batchDetails}.
    		\item Exceptions:
    		\begin{itemize}
    			\item InvalidBatchException: One of the following occurs:
    			\begin{enumerate}
    				\item The Customer Organization on whose behalf the batch would be submitted may not submit the specified type of documents.
    				\item If this concerns a recurring batch, the number of entries contained in the batch to be submitted exceeds the number of allowed entries.
    				\item If this concerns a recurring batch, the Customer Organization is trying to submit the batch too early.
    				\item If it is indicated this recurring batch contains corrections of entries the Customer Organization tried to submit earlier, there is no such earlier submission of a recurring batch.
    				\item If it is indicated this recurring batch contains corrections of entries the Customer Organization tried to submit earlier, the total number of entries submitted as part of that recurring batch would exceed the number of allowed entries.
    			\end{enumerate}
    		\end{itemize}
    	\end{itemize}
    	
    	\item \texttt{void submitRawData(RawDataBatch batch) throws NoCorrespondingPlannedBatchException, BatchVerificationException}
    	\begin{itemize}
    		\item Effect: Accepts a raw data batch, verifies the individual entries and generates jobs from them
    		\item Exceptions:
    		\begin{itemize}
    			\item NoCorrespondingPlannedBatchException: there was no call to \texttt{submitPlannedBatchDetails} with a \texttt{BatchDetails} object that corresponds to the given batch
    			\item BatchVerficationException: One of the following occurs:
    			\begin{enumerate}
    				\item The batch was improperly formatted (for example, the batch was a zipfile that could not be opened)
    				\item Some individual entries are incorrect; the thrown exception indicates which entries are incorrect and why. Note that all entries that did pass verification were processed correctly.
    			\end{enumerate} 
    		\end{itemize}
    	\end{itemize}
    \end{itemize}
\end{itemize}

\subsection{ZoomitDeliveryHandler}
\begin{itemize}
    \item \textbf{Description:} Responsibilities of the component.
    \item \textbf{Super-component:} The direct super-component, if any.
    \item \textbf{Sub-components:} the direct sub-components, if any.
\end{itemize}

\subsubsection*{Provided interfaces}
\begin{itemize}
    \item ZoomitDocumentDelivery
    \begin{itemize}
        \item \texttt{void checkServiceAvailability(ParamType param) throws SomeException}
        \begin{itemize}
            \item Effect: Describe the effect of the operation
            \item Exceptions:
            \begin{itemize}
                \item SomeException: Describe when the exception is thrown.
            \end{itemize}
        \end{itemize}

        \item \texttt{void operation2(ParamType2 param)}
		    \begin{itemize}
                \item Effect: Describe the effect of the operation
                \item Exceptions: None
            \end{itemize}
    \end{itemize}

    \item InterfaceB
    \begin{itemize}
        \item \texttt{returntType2 operation3()}
        \begin{itemize}
            \item Effect: Describe the effect of the operation
            \item Exceptions: None
        \end{itemize}
    \end{itemize}
\end{itemize}